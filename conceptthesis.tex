%\documentclass[12pt,a4paper,titlepage]{article}

%\parindent 0pt
%\parskip 2ex
%\bibliographystyle{apalike}
% 
%\usepackage{natbib}
%\usepackage{graphicx}
%\usepackage{subcaption}
%\usepackage{longtable}
%\usepackage[titletoc, title,header]{appendix}
%\usepackage{tabulary}
% 
%%\usepackage[latin1]{inputenc}
%\usepackage{amsmath}
%\usepackage{amsfonts}
%\usepackage{amssymb}
%\author{Merijn Slagter}
%\title{Sustainability of Paludiculture in the Netherlands \\ ESA Thesis}
%
%\usepackage{enumerate}
%\usepackage[colorlinks, allcolors=blue]{hyperref}
%\usepackage{tabularx} 
%\usepackage{indentfirst} 
%\usepackage[normalem]{ulem}
%
%\newcommand{\sur}[1]{\ensuremath{^{\textrm{#1}}}}
%\newcommand{\sous}[1]{\ensuremath{_{\textrm{#1}}}}
\documentclass[a4paper,12pt]{scrbook}
\usepackage[T1]{fontenc} % Include italic fonts
%\usepackage{fontspec} % Compile with \XeLaTeX
\usepackage{geometry} % Page margins
\usepackage{titling} % Title page container
\usepackage{wrapfig} % Picture container
\usepackage{graphicx} % Including graphics
\usepackage{natbib}
\usepackage{appendix}
\usepackage{tabularx} 
\usepackage{tabulary}
%\usepackage{indentfirst} 
%\usepackage[colorlinks, allcolors=blue,xetex]{hyperref}
\usepackage[colorlinks, allcolors=blue]{hyperref}
\usepackage{hyperxmp} % Import license metadata
\usepackage[acronym,toc,shortcuts,nohypertypes=acronym]{glossaries} % Acronyms
\usepackage{tikz} % Charts
\usepackage{subcaption} % Grid of figures
\renewcommand{\familydefault}{\sfdefault}
\bibliographystyle{apalike}
\newcommand{\sur}[1]{\ensuremath{^{\textrm{#1}}}}
\newcommand{\sous}[1]{\ensuremath{_{\textrm{#1}}}}
\usepackage[normalem]{ulem}
\renewcommand\textbullet{\ensuremath{\bullet}}

\title{Paludiculture as a sustainable option in the Netherlands to reduce greenhouse gas emissions from peat soils}
\author{Merijn Slagter}
\date{\today}
\hypersetup{
    pdflicenseurl={http://creativecommons.org/licenses/by-sa/4.0/},
    pdfcopyright={This work is licensed under the Creative Commons Attribution-ShareAlike 4.0 International License.},
    pdfauthor={\theauthor}, % These are supposed to be the default but don't seem to be
    pdftitle={\thetitle},
    pdflang={en-GB}
}

% Flowchart blocks
\usetikzlibrary{shadows,arrows,positioning,shapes}
\tikzstyle{block} = [rectangle, draw, fill=white, 
    text width=8em, text centered, rounded corners, minimum height=3em]
\tikzstyle{line} = [draw, -latex']
\tikzstyle{data} = [trapezium, trapezium left angle=70, trapezium right angle=110, text centered, draw,
    text width=7em, fill=gray!30, double copy shadow, minimum height=3em]
\tikzstyle{datasingle} = [trapezium, trapezium left angle=70, trapezium right angle=110, text centered, draw,
    text width=8em, fill=gray!30]
\tikzstyle{bigdata} = [trapezium, trapezium left angle=80, trapezium right angle=100, text centered, draw,
    text width=14em, fill=gray!30, double copy shadow, minimum height=3em]

% Define acronyms
\newcommand{\underletter}[1]{\textbf{#1}} % Quick toggle of underlines
\newacronym{WUR}{WUR}{\underletter{W}ageningen \underletter{U}niversity \& \underletter{R}esearch}
\newacronym{NA}{NA}{\underletter{N}ot \underletter{A}vailable}
\newacronym{QGIS}{QGIS}{\underletter{Q}uantum\underletter {G}\underletter{I}\underletter{S}}
\newacronym{CSV}{CSV}
{\underletter{C}omma \underletter{Se}perated \underletter{V}alue}
\newacronym{GeoTIFF}{GeoTIFF} {\underletter{G}eospatial \underletter{T}agged \underletter{I}mage  \underletter{F}ile \underletter{F}ormat}
\newacronym{N}{N}{\underletter{N}itrogen}
\newacronym{P}{P}{\underletter{P}hosporous}
\newacronym[longplural={greenhouse gasses}]{GHG}{GHG}{\underletter{g}reen\underletter{h}ouse \underletter{g}as}

\geometry{top=1.25cm,bottom=0.96cm,inner=2cm,outer=1.91cm,foot=0cm,includeheadfoot} % First page margins

\begin{document}
 \begin{titlingpage}
  {\Large Environmental Systems Analysis}\vspace{0.9cm}
  
  {\Large Thesis Report ESA-2017-xx}\vspace{0.9cm}
  
  \hrule\vspace{1.1cm}
  
  {\bfseries \Large \MakeUppercase{\thetitle}}\vspace{2.0cm}
  
%  \begin{wrapfigure}{r}{0.55\textwidth}
%    \vspace{1cm}
%    \includegraphics[height=9.5cm,draft]{figures/wurlogo.png}
%  \end{wrapfigure}
%  
  {\Large \theauthor}\vspace{5.5cm}
  
  \rotatebox{90}{\Large \thedate}\vspace{1.5cm}
  
  \sloppypar{\hspace{-2cm}\includegraphics[width=13cm]{figures/wurlogo}}
  
  \sloppypar{\noindent\makebox[\textwidth]{\hspace*{\dimexpr\evensidemargin-\oddsidemargin}\includegraphics[width=\paperwidth]{figures/wurlogo}}}
  
  \newgeometry{top=1.25cm,bottom=1.25cm,inner=1.91cm,outer=1.91cm,foot=1.19cm,includeheadfoot} % Subsequent page margins
  \thispagestyle{empty}
  
  \begin{center}
  {\bfseries \Large \thetitle}\vspace{2.7cm}
  
  {\Large \theauthor}\vspace{1.1cm}
  
  {Registration number 000000000}\vspace{3.5cm}
  
  {\large \underline{Supervisors}:}\vspace{1.1cm}
  
  {Dr Karen Fortuin}
  
  {Dr Rik Leemans}\vspace{3.0cm}
  
  {A thesis submitted in partial fulfilment of the degree of Master of Science}
  
  {at Wageningen University and Research Centre,}
  
  {The Netherlands.}\vspace{2.7cm}
  \end{center}
  
  \begin{flushright}
    {\thedate}
  
    {Wageningen, The Netherlands}
  \end{flushright}\vspace{0.5cm}

    Thesis code number: ESA -
  
    Thesis Report: ESA-2017-xx
  
    {Wageningen University and Research Centre}
  
    {Environmental Systems Analysis Chair Group}
\end{titlingpage}

\newgeometry{top=1.25cm,bottom=1.25cm,inner=1.91cm,outer=1.91cm,foot=1.19cm,includeheadfoot}

\chapter*{Abstract}

\textbf{Keywords:} 

\tableofcontents

\chapter{Introduction}

\section{Problem background}

% peatland definition and global peat situation
Peatlands are formed over time by the formation of layers of dead vegetation, which due to its wet and thus anoxic condition cannot be processed by micro-organisms \citep{clymo1984limits}. A peatland is defined as an area with a peat layer consisting of at least 30\% dead organic material \citep{joosten2002wise}. On all continents around the globe, peat areas can be found \citep{joosten2002wise}. In South-East Asia, mainly in Indonesia, tropical peat forest exist \citep{page2011global}. In Western Europe, peat formation started to form after the last glaciation, approximately 11.000 years BC \citep{brouns2016effects}. Moreover, big spans of peatlands covered by ice can be found in Siberia \citep{frey2005amplified}. Currently, 1.5 \% of the terrestrial surface are peat areas.

% exploitation of peat around the world
Worldwide, peat has been exploited or used for various purposes. Peatlands are e.g. drained in order to exercise economically viable land use practices such as agriculture and forestry. This happens all over the world. However, Europe has a much longer history of draining peat soils compared to for example Indonesia \citep{joosten2002wise}. A major use for peat is in horticulture industry. Since peat has very suitable characteristics as a plant substrate, large areas in Eastern Europe and Canada are being excavated to obtain the material. In the European Union, about 29 million m\sur{3} of peat per year is used to produce growing media \citep{blievernicht2012youngest}.

% carbon stock in peat and GHG emissions
As a result of peat extraction and utilization of peat areas, \ac{GHG} emissions are emitted to the atmosphere. It is estimated that one third of the global carbon stock is stored in peat areas \citep{page2011global}. Currently, approximately 2 Gigatons of \sous{2} equivalents are being emitted to the atmosphere from peatlands \citep{joosten2009global}. That is 6\% of all global \ac{GHG} emissions \citep{joosten2012peatlands}.

%Peat presence in the Netherland
The Netherlands are what we could call a typical 'peat' country. Untill approximately 1200 years ago, more than 40\% of the Dutch surface consisted of peat \citep{vos2015origin, de2008vergeten}. These peat areas came up to several meters above sea level \citep{de2008vergeten}. Currently, only 7\% of the Dutch land surface consist of peat areas \citep{de2004verbreiding}. 

%Peat usage in the Netherlands
Dutch peatlands have had various uses over time. Before 1500, peat areas were mainly exploited for arable farming \citep{ettema2005boeren}. After 1500, peat areas were found to wet for arable land use. Therefore, windmills were used to drain water away. Since the Dutch peat areas where rather wet, the main land use became livestock farming \citep{brouns2016effects}. Peat was furthermore used as a fuel source \citep{van1996turfwinning}. The abundance and proximity of peat as a fuel source to Dutch cities in the middle ages was, according to some authors, part of the success story of the Dutch 17\sur{th} century Golden Age \citep{dezeeuw}.

%Current drainage and GHG emissions from Dutch peatlands
Currently, most of the Dutch peatland areas in the Netherlands are drained with pump installations \citep{brouns2016effects}. As a result, Dutch peatlands have subsided upto a couple of meters \citep{de2008vergeten}. \ac{GHG} emissions from peatland subsidation make up 2.5 \% of the total GHG emissions \citep{van2010emission}. 

%Cause of GHG emissions from peat
The actual cause of \ac{GHG} emissions from drained peatlands is the process of peat oxidation. This process occurs when peat is aerated. When oxygen can enter the peat, micro organisms can decompose the organic matter that is stored in the peat through a redox reaction with oxygen \citep{erkens2016double}. The subsidization due to oxidation is what \citet{kuntze1984bewirtschaftung} called the vicious cycle of peatland utilization and is portrayed in Figure \ref{fig:peatlanduti}.

\begin{figure}
    \centering
    \includegraphics[scale=0.6]{figures/Kuntze1984} 
    \caption{Vicious cycle of peatland uitilization}
    \label{fig:peatlanduti}
\end{figure}

%Mitigation measures
In order to reduce peatland subsidation in the Netherlands, different mitigation measures are possible. All of these measures consist of heightening the water table to a certain degree. Currently, the technology of underwater drainage is proposed as a solution for a large part of Dutch peatlands. With this apparatus, water is provided to the peatlands in the summer, the season in which most oxidation takes place due to dry circumstances \citep{van2011huidige}. Another option is to transform the peatlands to nature conservation areas by rewetting the degraded peatlands. Rewetting degraded peatlands and using it for wet crops is also an option that is currently thoroughly investatigated \citep{Wichmann20151063}. Wet agriculture while protecting organic soils is called paludiculture \citep{joosten2002wise}. 

%Awareness about paludiculture is an option to stop peat decline
Various authors have suggested that only with rewetting peatlands, peat decline can be fully stopped \citep{van2016dalende, wichtmann2016paludiculture}. Paludiculture is the only alternative for production farming instead of livestock farming on drained soils. It is therefore not suprising that awareness among various scholars and policy makers is rising that paludiculture will be a significant part of the future land use on Dutch peatlands \citep{abel2013database, wichtmann2016paludiculture, Wichmann20151063}. 

% Categories and varieties of paludiculture
Different categories of paludiculture are consumption crops, litter, building materials, medicines and biomass for energy \citep{wichtmann2016paludiculture}. A large potential is currently seen in farming peat moss (Spaghnum) for use in horticulture and reed and cattail as building materials \citep{wichtmann2016paludiculture}.

\section{Problem statement}
%GHG emissions from peat soils if drained.
In deep-drained Dutch rural peat areas water levels can become between -10 and -100 cm below soil level in summer. As a result, the peat is aerated and can then oxidize. With the oxidation, \acrlongpl{GHG} are emitted to the atmosphere and the soil subsides, causing infrastructural damage. The oxidation can be reduced by fixating the water level to a level constantly above the peat. Such a high water level in peat lands is unsuitable for other agricultural purposes than wet agricultural crops (paludiculture). 

%conflicting interests regarding land use in Dutch peat meadows
However, a transition to paludiculture is not easily achieved since there are many interests regarding land use in Dutch peat meadows. For example, farmers have invested in large sheds and lands and meadow birds use the agricultural lands for breeding and foraging. Moreover, the question arises whether paludiculture can be a economic viable alternative to current dairy farming. Because of these conflicting interests, it is currently unknown whether and where paludiculture is a sustainable option to reduce peat mineralization in the Netherlands. 

\section{Goal and research questions}
The goal of this study is to give insight in the (spatial) opportunities of paludiculture as a sustainable option to minimize peat mineralization in Dutch rural peat areas. In order to reach the goal, the following research questions and sub research questions were defined:

\begin{enumerate}
\item \label{rq:pallit} {Can paludiculture help to reduce peat mineralization in the Netherlands?}
\item \label{rq:abiot} {Where are abiotic requirements of three paludiculture species met by Dutch abiotic conditions?}
\item \label{rq:gisoptloc} {What are optimal locations to reduce peat mineralization in the Netherlands?}
\begin{enumerate}
\item \label{rqsub:peatmineral} {Where and how much peat mineralization is taking place in the Netherlands?}
\item \label{rqsub:catchpeat} {What catchments can be used best to be completely rewetted?}
\item \label{rqsub:catchheigth} {Does rewetting lower parts of a catchment help to reduce peat mineralization?}
\end{enumerate}
\label{rq:sust}\item{Which economic, ecological, social and cultural opportunities and bottlenecks need to be accounted for when implementing paludiculture in the Netherlands?}
\end{enumerate}

\chapter{Data and methods}

In order to answer the research questions, literature studies, several spatial analyses and a participatory study have been performed. In Table .., the methods used for each research question is shown. 

\section{Study area}

%Map of Dutch peat soils
This thesis focused on paludiculture in (organic) peat soils in the Netherlands. 

\section{Data}
\subsection{Spatial data}
\subsubsection{Data availability}
For this thesis, data was partially obtained as free open source data from the web. Other data was provided for by Dutch Waterschappen and Wageningen Environmental Research. Table \ref{tab:spatialdatasources} shows the sources and years of creation of the spatial data I used. 

\subsubsection{Data processing mechanisms}
For this thesis, the open source programming language R and the open source programs \ac{QGIS} and RKWard were used. RKWard makes use of R and was in this thesis used for all the data processing. \ac{QGIS} was used to derive the visualization products.

\subsubsection{Data preparation}
%File formats
The different data sets were provided in different formats and were not always directly suitable for the intended usage. For spatial vector data ESRI Shapefiles (from now on called shapefiles), for spatial raster data \ac{GeoTIFF} files and for tabular information \ac{CSV} files were used. These formats are easy in use and can easily be integrated with the used data processing mechanisms (i.e. \ac{QGIS} and R). Some data was provided as geodatabase, and had thus to be converted to shapefile format. In this thesis, many different data sources were integrated. In order for the data to be compatible, the same projection systems had to be used. However, not for all calculations, the same projection systems are suitable. Therefore, for this thesis, two projection systems were used, namely:

% Projection system used:
\begin{itemize}
\item Amersfoort New: 
%+proj=sterea +lat_0=52.15616055555555 +lon_0=5.38763888888889 +k=0.9999079 +x_0=155000 +y_0=463000 +ellps=bessel +units=m +no_defs
\item WGS84: 
%+proj=longlat +datum=WGS84 +no_defs +ellps=WGS84 +towgs84=0,0,0
\end{itemize}

Table \ref{tab:procdataprep} lists the preparation procedure of the different source files prior to the data analysis. Annex \ref{ann:procdataprep} shows the r scripts and the detailed description of the data preparation procedure.

\subsection{Peat mineralization indices}
% Stouthamer indices
As part of the answer to research question \ref{rqsub:peatmineral}, information on peat mineralization occurence in the Netherlands was needed. Peat soils differ in their mineralization potential. Therefore, mineralization potential indices can be calculated. Some of these indices  were produced by \citep{stouthamer2008toelichting}. An overview of these indices can be found in Annex .. These indices were created based on a four criteria: (1) soil type of the layers, (2) depth of the layers, (3) organic matter content in the layers and (4) profile type of the layers. A detailed description can be found in \citep{stouthamer2008toelichting}. 

\section{Spatial data analysis}

\subsection{Abiotic conditions in the Netherlands}
In order to get an understanding of the abiotic situation in the Netherlands regarding paludiculture plants, measurements of pH, \ac{N} and \ac{P} that were done in peat areas in the Netherlands were summarized. As a result, suitable locations for three paludiculture species could be depicted. In Table \ref{tab:workflowabiotic}, a workflow is provided. In Annex \ref{ann:procedureabiotics}, a more detailed approach is provided, as well as the r scripts I used to get the analysis done.

\subsection{Peat mineralization reduction potential in the Netherlands}

\subsubsection{Mineralization potential indices of peat soils in the Netherlands}

%Deriving mineralization potentials
In Table \ref{tab:worklowminerpot}, an overview of the workflow to derive mineralization potentials is shown. \citet{stouthamer2008toelichting} calculated potentials for .. peat types. In the Netherlands, .. peat types can be found. I used two approaches to update the mineralization potentials to the residing peat types. First of all, based on the already existing mineralization potentials, I could extrapolate the indices to ... other indices. However, for .. peat types, this could not be done. Therefore, for these peat types, indices were created from scratch, using the method used by \citet{stouthamer2008toelichting}. In Annex .., a description of this method is showed. Finally, for .. of the .. peat types, indices were found, as listed in Annex .. 

%Categorize indices
In order to summarize the mineralization potentials, the indices were categorized in four groups of mineralization potential: Very low, Low, High and Very high. Table \ref{tab:minpotcatranges} shows the ranges on which these categories are based. Subsequently, the areas of these categories per province and for the Netherlands were calculated. Annex .. shows the exact procedure that was followed to do the analysis.

\subsubsection{Catchments that can be used best to be completely rewetted}

%Workflow to determine catchments that can be used to be completely rewetted
To answer Research question \ref{rqsub:catchpeat}: "What catchments can be used best to be completely rewetted?", first of all, the catchments that contain peat soil were determined. Secondly, the proportion of peat soil in each catchment was calculated. The proportion was subsequently categorized into four levels of peat presence: Very low, Low, High and Very High. As a result, per waterschap, the proportional presence of peat in the catchments could be presented. An overview of the performed steps is depicted in Table \ref{tab:workflowcatchmentrewet}. A more detailed description can be found in Annex .., as well as the r scripts that were used to perform this analysis.

\subsubsection{Height differences within catchments}
Reserach question \ref{rqsub:catchheigth} addresses the question: "Does rewetting lower parts of a catchment help to reduce peat mineralization?". In order to answer this question, the height difference between the peat part and the non peat part of .. catchments was determined. These catchments were selected based on practical reasons. An overview of the procedure can be found in Table \ref{tab:workflowheightdifcatch}. Annex .. shows the detailed process as well as the r scripts that were used to do the analysis.

\subsection{Sustainability issues}

\subsubsection{Nature areas in the Netherlands}

\subsubsection{Important areas for meadow birds}


\section{conference and interviews}

\chapter{Peat mineralization reduction by paludiculture}

This chapter addresses the question: "Can paludiculture help to reduce peat mineralization in the Netherlands?” As a first step, the concept of paludiculture will be explained and various promising species will be described. Then, a literature study will be presented about the greenhouse gas balance of rewetted and paludiculture plots compared to drained peat soils.

\section{Promising species of paludiculture}

\subsection{The concept of paludiculture}
%Paludiculture definition
Paludiculture stems from the Latin word Palus, which means swamp. The term paludiculture was coined in 2007 and basically describes the productive usage of wetlands while reducing greenhouse gas emissions from peat areas \citep{wichtmann2007paludiculture}. 

%Number of suitable plant species for paludiculture in the Netherlands
A typical Dutch paludiculture crop is Common Reed, which is a water plant species used for thatching roofs \citep{wichtmann2016paludiculture}. Only recently, \citet{abel2013database} selected a list of 812 water plants that are globally suitable for cultivation. In their research, 184 plants were found suitable for Western Pomaria in Germany. Since the climate in this region is similar to the Dutch climate, 184 plants can be classified suitable to cultivate in the Netherlands. 
%Ask Susanne to provide a list

% Current paludiculture practices
Currently, some paludiculture practices already exist in the Netherlands. At least 8 million kilo Common Reed is harvested every year in the Netherlands. Usage is even higher, with at least 20 million kilo a year \citep{wichtmann2016paludiculture}. Also, the island of Terschelling is known to have cranberry plantations. Cattail and peat moss are two other common species in Dutch ecosystems that have been put forward as very promising species to cultivate \citep{abel2013database, van2013werk}. Also, Azolla, which is not native to the Netherlands, is mentioned as a promising species for paludiculture \citep{abel2013database, van2013werk}. 
% Maybe add the description of species in boxes or annex?
These three species will be elaborated on below.

\subsection{Cattail}

%Find own cattail picture
\begin{figure}
    \centering
    \includegraphics[scale=0.2]{figures/cattailsmall} 
    \caption{Cattail, \textit{Typha}}
    \label{fig:cattail}
\end{figure}

% Determine whether to use latin names or English names.

%Difference between latifolia and angustifolia
\textit{Typha} or cattail is a plant genus endemic in Europe (see Figure \ref{fig:cattail}). Two prominent species of this genus are \textit{Typha latifolia} or broadleaf cattail and \textit{Typha angustifolia} or narrowleaf cattail. Both species were studied as a potential crop in prior agricultural plots in Germany in 1998 \citep{wild2001cultivation}. Both species were planted and its sprouts multiplied with a factor 7-10 in two months. The cattail family is known as a robust plant which can easily tolerate flooding and high nutrient contents \citep{wild2001cultivation}. \citet{heinz2011population} compared both cattail species and found a number of major differences between the species. First of all, broadleaf cattail produces ten times more offspring from shoots in optimal conditions suggesting it to be a very pioneering plant. Narrowleaf cattail is more adapted to a competitive strategy, the species has for instance a higher salt-tolerance \citep{heinz2011population}. A study by \citet{grace1982niche} showed that narrowleaf cattail can grow better in deeper water than common cattail due to its greater leaf surface area.

%Experiments 
Experiments with cattail are taking place in Germany since the '00s, but have only recently started in the Netherlands. In Zegveld, a pilot of 0.5 hectares with broadleaf cattail started in 2012$Check date$. In 'Het B\^utefjild', another pilot project was started. 

%Usage of cattail
Although usage of cattail is currently limited, it was used in the past in the Netherlands. \citet{morton1975cattails} notes that it was used to stuff chairs and make winter protection for crops. Although cattail is hardly used nowadays, interest in cattail as a paludiculture crop is growing. Various authors have suggested cattail as a promising species \citep{morton1975cattails, heinz2011population}. They mention cattail promising, not only for its robustness, but also for its potential uses. Cattail has many purposes, which are listed in Table \ref{tab:typha}. The most promising usage is as isolation material. Due to its fire-resistant properties, it is a fire-proof alternative for isolation. It can be made into isolation boards or as a wall-filler. 

% Include Morton (1975)

\begin{table}
\centering
\caption{Possible usage of Typha}
\begin{tabular}{|c|c|}
\hline 
\textbf{Usage} & \textbf{Part} \\ 
\hline 
Isolation & Whole \\ 
\hline 
Bio-laminate & Whole \\ 
\hline 
Fodder & Whole \\ 
\hline 
Material for handworking & Stem \\ 
\hline 
Human Food & Shoots, Rhizomes and Cobs \\ 
\hline 
Animal Fodder & Shoots, Rhizomes and Cobs \\ 
\hline 
Flour & Flower \\ 
\hline 
Medicinal tea & Flower \\ 
\hline 
Organic pest control of mites & Flower \\ 
\hline 
• & • \\ 
\hline 
\end{tabular} 
\label{tab:typha}
\end{table}

\subsubsection{Peat moss}

\begin{figure}
    \centering
    \includegraphics[scale=0.5]{figures/peatmosssmall} 
    \caption{Peat moss, \textit{Sphagnum}}
    \label{fig:peatmoss}
\end{figure}

% peat moss in the Netherlands
Another plant that is put forward as promising is \textit{Sphagnum} or peat moss (see Figure \ref{fig:peatmoss}). Peat moss is a native species to the Netherlands that once dominated peat bogs in the Netherlands. The water supply in its natural condition comes solely from rain water. Experiments with peat moss started already in the 2001 in Germany \citep{gaudig2014sphagnum}. In 2013, a pilot project was started in 'Het Ilperveld', near Amsterdam \citep{van2013werk}. 

% alternative to horticulture substrate
Cultivation of peat moss is interesting since it is seen as a sustainable alternative to harvested peat for horticulture substrates. Currently, the best substrate for horticulture industry is still peat harvested from bogs. Every year, around 20 million m\textsuperscript{3} of peat is harvested in Europe to supply peat for horticulture \citep{altmann2008socio}. The highest quality is so-called white peat which has formed from sphagnum mosses over the last 3000 years \citep{gaudig2014sphagnum}. Peat moss farming is a good alternative to obtain substrates for horticulture. Using biomass from peat moss farming in horticulture works equally well or better than standard peat-based media. That is not surprising, since both alternatives have similar physiological qualities \citep{gaudig2014sphagnum}. 

% promosing peat moss species
According to \citet{gaudig2014sphagnum} and \citet{wichtmann2016paludiculture}, \textit{Sphagnum palustre} is one of the most promising species of the peat moss genus since it is highly productive. The authors furthermore mention that peat moss farming is already economically profitable. That is mostly because selling peat moss in niche markets results in high revenues. However, peat moss farming cannot yet compete with cheaply extracted white peat \citep{gaudig2014sphagnum}.

\subsection{Azolla}

\begin{figure}
    \centering
    \begin{subfigure}[a]{0.3\textwidth}
    	\includegraphics[scale=0.2]{figures/azollasmall} 
    	\caption{ \textit{Azolla} fern}
    	\label{fig:azollasmall}
    \end{subfigure}	
    \quad
    \begin{subfigure}[a]{0.3\textwidth}
    	\includegraphics[scale=0.11]{figures/azollafield} 
    	\caption{ \textit{Azolla} field}
    	\label{fig:azollafield}
    \end{subfigure}	
    \caption{ \textit{Azolla} \citep{wagner1997azolla}}
    \label{fig:azolla}
\end{figure}

% Azolla introduction
Another species that is put forward as interesting is \textit{Azolla}, a fast growing surface water plant very rich in proteins. The plant forms a symbiosis with \textit{Anabaena azollae}, a bacteria that fixes nitrogen from the air \citep{wagner1997azolla}. \textit{Azolla} is therefore never limited by nitrogen. \textit{Azolla} is not native in The Netherlands, but currently widely common in Dutch waters. \textit{Azolla} can be harvested for various purposes. Since a long time, \textit{Azolla} is used in Asian countries as green manure in rice paddies \citep{wagner1997azolla}. A promising use for the Netherlands is therefore also to use it as fertilizer. Potential purposes are mentioned by \citet{wagner1997azolla} and listed in Table \ref{tab:azolla}.

\begin{table}
\centering
\caption{Potential uses of \textit{Azolla}}
\begin{tabular}{|c|}
\hline 
\textbf{Potential Use} \\ 
\hline 
Fodder for animals \\ 
\hline 
Human consumption \\ 
\hline 
Medicines \\ 
\hline 
Production of biogas \\ 
\hline 
Production of hydrogen \\ 
\hline 
Water purifier \\ 
\hline 
Control of weeds \\ 
\hline 
\end{tabular} 
\label{tab:azolla}
\end{table}

\section{Greenhouse gas emission reduction by paludiculture practices}

\subsection{Effect of rewetting peatlands on peat oxidation}

% impact of rewetted peatland on GHG emissions
\citet{van2013rewetting} studied the effect of a rewetted peatland on net GHG emissions expressed in CO\sous{2} equivalents. The study showed that CO\sous{2} emissions decreased drastically and CH\sous{4} emissions increased. However, the net effect of rewetting the peatland with water table of 4 cm above the ground on emissions of CO\sous{2} equivalents was positive with approximately 1.0 kg m\sur{-2} less CO\sous{2} equivalents. Still, GHG emissions in terms of CO\sous{2} were as high as 2.0 - 3.0 kg m\sur{-2} in the rewetted condition. An experiment with top soil removal in the Netherlands, showed a reduction of 99\% in methane emissions when rewetting the peatland \citep{harpenslager2015rewetting}. The results shows the potential of topsoil removal prior to rewetting for reducing GHG emissions. 

\subsection{Potential of paludiculture crops to conserve peat}
% peat moss carbon balances
In Germany, carbon balances of establishing peat moss plots were made \citep{gunther2017greenhouse}. The authors found that methane emissions declined significantly over the study time of two years. The production strips of the peat moss  species were found to be net GHG sinks of 5–9 ton ha\sur{-1} a\sur{-1} (in CO\sous{2} equivalents). However, ditches were net carbon sources with release of about 11 ton ha\sur{-1} a\sur{-1} CO\sous{2} equivalents. Still, peat moss farming on bogs were found to be less emitting compared to low-intensity dairy farming. Moreover, the authors note that with peat moss farming, an alternative is provided for mineralisation of pristine peat areas for horticulture substrates.
 
% cattail carbon balances
It is known that plants possessing aerenchuma such as cattail, are methane emitters \citep{wichtmann2016paludiculture}. A study on broadleaf cattail showed that the GHG balance after the first year of implementation was already close to climate neutral \citep{guntherghgtypha}. In the study, the authors compared a plot with harvest with a plot without harvest. Harvesting the cattail proved to reduce net GHG emissions slightly. \citet{wichtmann2016paludiculture} note that 'renewed biomass growth after cutting supports net carbon sequestration contributing to a decrease in GHG emissions'. 

No studies have analyzed the effect of Azolla farming on the GHG balance. 

\chapter{Abiotic requirements of implementing paludiculture in the Netherlands}

\section{Abiotic requirements of three paludiculture species}

This chapter addresses research question \ref{rq:abiot}: "Where are abiotic requirements of three paludiculture species met by Dutch abiotic conditions?". First of all, the abiotic requirements of three paludiculture crops will be discussed. An overview of these requirements can be found in Table \ref{tab:phy}.

\begin{table}[htbp]
\caption{Abiotic Requirements of broadleaf cattail, peat moss and Azolla}
\begin{flushleft}
\begin{tabular}{|r|l|l|l|l|}
\hline
\multicolumn{1}{|l|}{} & \textbf{Factor} & \textbf{Cattail} & \textbf{Peat moss} & \textbf{Azolla} \\ \hline
1 & \textit{N presence} &  Higher is better & Higher is better & Does not matter \\ \hline
2 & \textit{P presence} & May be high & May be high, if subconditions are met & Must be high \\ \hline
3 & \textit{pH} & 4-8 & 3.0 to 7.0, if subconditions are met & 4.5 to 7.5 \\ \hline
4 & \textit{Water level} & -20 to +60 cm & -10 to 0 cm & + 5 cm \\ \hline
\end{tabular}
\end{flushleft}
\label{tab:phy}
\end{table}


\subsection{Cattail}

 %\citet{morton1975cattails} notes that seedlings of Common Cattail will die out if the water level is maintained at 45 cm from spring to midsummer. 

% water level
Cattail species are robust plants, but have some requirements for growing. Growth of mature plants of broadleaf cattail will be severely impaired by water depths of 63.5 cm \citep{morton1975cattails}. Narrowleaf cattail can survive in water depths up to 1.2 meters \citep{morton1975cattails}. \citet{dubbe1988production} also note that Cattail species can deal with water depths upto 50 and 115 cm above the ground. However, another study showed that the optimal water depth of broadleaf cattail is 22 cm above the ground \citep{grace1989effects}.
%What about minimum water levels?

% nutrients
\citet{ciria2005Typha} studied the potential of common cattail in waste water treatment. The authors found that common cattail did not have a problem with high concentrations of nitrogen (52 mg per liter) and phosphorous (23 mg per liter). Also, \citet{newman1996effects} found that adding 0.05 millilgram per liter P or 0.1 milligram per liter N was fine for broadleaf cattail growth.
% Provide a study that tells that higher N presence leads to more biomass production. If there is not enough, extra N can be fertilized.

% acidity
Common cattail grows well under acidic conditions ranging from a pH 5 to 7, but encounters problems if pH goes towards 3.5 \citep{brix2002typha}. Narrowleaf cattail seems to be able to deal with more extreme situations such as alkaline and saline situations. A pH of 9 can still be fine for Narrowleaf cattail.

% Maybe add extra column to table for difference between cattail species.

%Evapotranspiration can be between 105-150 cm per year, as reported in Poland \citep{dubbe1988production}.

\subsection{Peat moss}

% water level
In contrary to cattail, peat moss is a vulnerable species. Physical conditions should be within a delicate range. Peat moss grows best with the water level just below surface.  \citet{fritz2014paludicultuur} note that to optimize peat moss growth, the water table should be kept constant at 2-5 cm below ground level. However, as \citet{gaudig2014sphagnum} mention, it is also possible to grow peat moss on floating mats on the water.

% nutrients
\citet{Temmink2017196} show that peat moss grows well under high N input as long as pH is low enough and P and K inputs are high enough. 
% Note why that is important
N input was noted as \~ 30 kg per hectare per year and P was noted as between 1 and 3 kg per hectare per year.$How much is that in mg/l$ The authors furthermore note that high nutrients can cause competitor vascular plant weeds such as \textit{Juncus effusus} or algae to grow. However, in this study these plants were removed, which proved useful for the peat moss to grow. \citep{wichtmann2016paludiculture} note that peat mosses in natural bogs grows best with low nutrient contents.


\citet{Temmink2017196} note that pH should be kept low, which can be done by making sure bicarbonate concentrations in waters are low enough. In the experiment, ditch water pH concentrations were between 4.8 and 6.0. Also, in other studies it is mentioned that peat moss grows best in acidic conditions \citep{wichtmann2016paludiculture}. Also, \citet{gaudig2005growing} found that peat moss performed significantly better at a pH of 3.2 compared to 4.5 and 8.0. 

\citet{fritz2014paludicultuur} note that it would be better to remove the top layer of a previously used agricultural peatland, so the trophic conditions are better accustomed to peat moss growth. That this procedure is ideal for peat moss growth was also mentioned by \citet{Temmink2017196}.

%What is the precise perfect stoichiometry?
\subsection{Azolla}

\textit{Azolla} is a water plant that needs full water cover. It can only survive in dry fields for a few days \citep{wagner1997azolla}. The optimal water depth is 5 cm, although it will survive deeper depths \citep{wagner1997azolla}. \citet{sabetraftar2013review} revealed that Azolla plants can only grow when it is floating on water. They furthermore say that Azolla plants can dry out if humidity is below 60\% but that Azolla can grow well between 55 and 83 \% humidity.  Waves and winds could be a cause for death of Azolla, since it would break the fronds (leaves). Cattail is noted as a protector of Azolla for wind and waves \citep{sabetraftar2013review}.  

\citet{lumpkin1980azolla} note that the most common nutrient limiting Azolla growth is Phosporous. The authors note that in Denmark, Azolla was found to thrive with a P concentration of 1.1 mg P/liter. Also, Iron (Fe) can become limiting, but in general 1 ppm iron is  sufficient for Azolla growth. However, the authors note that iron deficiency might stem from a high pH, that is because iron can then be precipitated. They acknowledge, that the more P is put into the water, the more Nitrogen can be fixated. \citet{sabetraftar2013review} note that Azolla is known to not be affected by high N presence. They note that in laboratory experiments, P concentrations of 0.06 ppm  is sufficient. But from the field, they note P requirements of 0.3 - 1 ppm. Azolla plants can survice in salt concentrations between 160-380 mg/l  \citep{lumpkin1980azolla}. However, it was studied that high salinity can inhibit growth \citep{sabetraftar2013review, lumpkin1980azolla}. 

The optimum pH range is from 4.0 - 4.5, but Azolla can also survive in water with a pH from 3.5 - 10.0. However, vulnerability to pH is much affected by other factors. For instance, with high light intensity, optimal pH is 9-10 but with low light intensity, optimum pH is 6.0 at a temperature of 20 degrees \citep{wagner1997azolla}. Optimal pH levels according to \citet{lumpkin1980azolla} are in a range from 4.5 - 7.0. \citet{sabetraftar2013review} note that, in general, a pH between 4.5 and 7.5 is sufficient.


%According to Tim Pelsma, a researcher at Waternet Amsterdam, Azolla plants need a lot of phosphorous, more than can be provided by Dutch water systems. Therefore, extra P fertilization is necessary for Azolla growth (interview...).


%However, high  nitrogen content in water can affect Azolla growth.

\section{Dutch abiotic conditions}

In the Netherlands, surface water quality is measured in many places. Based on data from 2013, a number of parameters could be mapped. An overview of the parameters, the number of measurement locations per parameter and the number of locations in peat areas can be found in table ... 

In table .., the number of locations that meets the conditions of each paludiculture crop can be found. Moreover, Figure .. presents per waterschap the suitability of locations for each crop. 

%\citet{van2010nutrienten} studied the current status of nutrients in Dutch fresh waters. They calculate an average N concentration of 3.56 milligram per liter and average P concentration of 0.52 milligram per liter. 

\chapter{Paludiculture and peat conservation}
% In dit hoofdstuk ga ik in op aspecten die handelen met veen oxidatie. eerst kijk ik naar (1) wat is veen oxidatie, hoe gebeurt het, welke eenheden worden gehanteerd, (2) hoe kan water ophoging en paludicultuur volgens de literatuurl veen oxidatie verminderen, (3) welke veentypes zijn meer/minder kwetsbaar voor veenoxidatie en waar in Nederland kunnen we deze gronden vinden? (4) waar worden de dikste veenpakketten in Nederland gevonden en waar kunnen we dus verwachten dat het meeste veen zal oxideren in de toekomst als we doorgaan met ontwateren? (5) vervolgens kijken we naar wat de invloed is van water management strategiien op veen oxidatie in gebieden. Dit is meer een theoretische exercitie met veel aannames omdat er nog niet zo veel onderzoek naar gedaan is. (6)  Als laatste kijken we dan naar natuurgebieden die in het veen liggen. Door wegzijging naar omliggende gebieden worden natuurgebieden vaak te droog. Hoe werkt dit en is het nuttig om om deze natuurgebieden heen paludicultuur plantages aan te leggen? Waar in Nederland is dit het geval?  

In this chapter, the research question "what are optimal locations to reduce peat mineralization in the Netherlands?" will be addressed.

Not all peat areas are the same. Peat areas can differ in composition. For example, in some areas pure peat layers of a few meters exist whereas in other areas, on top or below the peat, mineral layers can be found. Through a different organic matter content in different peat compositions, the potential of the peat to mineralize and thus emit greenhouse gas emissions is also different. For the Netherlands, in the first section of this chapter, I mapped areas with different mineralization potentials.

A second and third analysis focuses on the water management in the Netherlands. That is, water levels in the Netherlands are regulated per catchment area. If we take the current catchment system into account, we could use a whole catchment for paludiculture crops. If the goal is to reduce GHG emissions, it is most optimal to use catchments with the most of its area covered by peat. The second analysis will focus on that.

The third analysis will subsequently look into using parts of catchments for paludiculture. If we want to reduce GHG emissions, the most optimal to do is to use the peaty part of a catchment for paludiculture. Of course, these peaty parts must lie lower than the non peaty parts. Otherwise, the part of the catchment cannot be rewetted. Therefore, the third analysis will look into a number of catchments and check the height difference between the peaty and the non peaty part of the catchment.

\section{peat mineralization potential of Dutch peatlands}

In this section, I will provide insight in the potential of the Dutch peatlands to mineralize. This potential is based on the difference in soil structure of the different peatlands. First of all, peatlands can be called peatlands when the soil contains more than 20 \% organic matter \citep{}. 

\subsection{Distribution of peatlands in the Netherlands}

The results of my study will be presented here. Overall, I found that peatlands exists in all provinces of the Netherlands. However, the peat types do differ. 
$In general, we can say that the North Western part contains fens and in the South Eastern part bogs can be found. However, to be more specific, Table .. shows the distribution of the main peat types over the different provinces in the Netherlands.$

\subsection{deriving peat mineralization potentials}
For ... peat soils in the Netherlands, mineralization potentials were calculated by ...  In this study, I extended the list of mineralization potentials to include also the other peat types in the Netherlands. The results are presented in Annex .. Furthermore, I categorized the different peat types in 4 categories of peat mineralization potential, which can be seen in table ..

\subsection{spatial distribution of peat mineralization potentials}
For the whole of the Netherlands, peat mineralization potentials were mapped, this map can be found in Annex .. Table .. shows the area size of the different mineralization categories in the Netherlands. Furthermore, in Figure .. a map can be found portraying per province the area size of each mineralization category. The areas with categories of mineralization potentials are summarized in table ..

\section{catchment areas in peat zones}

\subsection{catchments with peat}
For 5 waterschappen, I obtained information of borders of catchment areas. These borders were merged into one map. A calculation was made to obtain the catchment areas that partially or completely consists out of peat areas. For the 5 waterschappen, I found ... catchments that have some peat soil. Table .. lists per waterschap, the number of catchments that have some peat soil. These catchments are displayed in Annex .. 

\subsection{amount of peat in catchment}
Moreover, I analyzed the percentage of peat soils versus non peat soils in each remaining catchment. Following, these percentages were categorized in four groups. For the 5 waterschappen, the results are shown in table .. Also the contribution of each category to the total catchments with peat is shown. Figure .. displays per waterschap, the number of peaty catchments and the contribution of each category to the total amount of catchments. 

\section{height differences in peat catchments}
For .. catchments, it was calculated what the average height difference was between the peat and the non peat part of the catchment. The names, the sizes and the amount of peat in the catchment can be found in Table ...Annex .. shows where in the Netherlands these catchments are located.

\subsection{statistics of the catchments}
For the catchments, boxplots of height were calculated in order to tell something about the actual height distribution within the catchment and can be seen in Figure .... 

\subsection{height differences between peat and non peat part of catchments} 
For all catchments, mean heights were calculated, as well as heights of the peaty and non peaty part of the catchment. Together with the difference, the results are presented in Table .. Moreover, boxplots were made where we can see heights for the peaty versus the non peaty parts.

% main question: what sustainability bottlenecks and opportunities need to be accounted for? 
% sub question: what social, economic and ecological bottlenecks and opportunities are there?
% sub question: what spatial and temporal bottlenecks and opportunities are there?

\chapter{issues of sustainability}

\section{Social, Economics and Ecological aspects}

\subsection{Economic factors}
% incomes of farmers versus paludiculture
With intensive dairy farming, the net income of a farmer is 565 euro per year \citep{riet2013}. On average, an intensive dairy farmer receives 400 euros of subsidies from the governments per hectare per year \citep{riet2013}. At the moment, one of the most profitable paludiculture crops is peat moss, with which the net income is calculated as 2.650 euro per hectare per year \citep{riet2013}. However, a first investment of 23.300 euro would be needed \citep{riet2013}. No subsidies are given to paludiculture farmers. Note that land prices of agricultural lands in peatareas are between 50.000 and 60.000 euro per hectare \citep{wur}.

% accounting for other ecosystem services
If ecosystem services costs and benefits would be accounted for, paludiculture would be more economically viable. For example, with a functioning carbon credits system of 35 euro per ton reduced GHG emissions, a peat moss farmer could earn 466.67 euros extera per hectare per year. Moreover, \citet{born} calculated that costs of damage to buildings and infrastructure as a result of peat decline will be in total 1 billion euros till 2050, which would be ~ 30 million euros per year. If other ecosystem services which are presented below would be accounted for, benefits of paludiculture would be higher.

\subsection{Ecological factors}
% Low lying nature conservation
In the Netherlands, a number of nature conservation areas lie in peatlands. These areas often have a higher elevation than their surroundings. That causes seepage from water away from the nature area, resulting in drougths and high costs to keep the water in. Paludiculture as buffers around nature areas are mentioned as a good alternative to prevent seepage from nature areas, and also to function as a bufferzone for aquatic fauna species such as the aquatic warbler \citep{riet2014}. 

% Water storage after extreme weather circumstances
Besides acting as a nature bufferzone, paludiculture areas can be used as water storage bufferzones, in case of extreme rainfall, which is expected to happen more frequently in the future. 

% Nutrient filtering
Surface waters in the Netherlands are often eutrophied due to nutrient runoff after intensive fertilizer and manure application in agricultural fields. In special, N and P are abundently available in surface waters, causing algae and other plants to bloom. As a result, waters can be deprived of oxygen, resulting in a (undesirable) change of aquatic fauna /citep{Waajen2014}. Paludiculture crops have shown to be able to thrive under these high nutrient conditions (see chapter \ref{chap:abiot). Waters can be cleaned by paludiculture crops, which is already done in waste water treatment plants \citep{Ciria2005;Merje2009}. Depending on the nutrient concentrations, wetland species can extract more than 300 kg N and 30 kg P per hectare per year \citep{land2016}. 

% Meadow birds
A topic that deserves special attention is the declining meadow bird populations in the Netherlands. Some of these species, such as the Godwid, has its main habitat in the Netherlands. Therefore, the Netherlands has a special opportunity to protect these species. The main reason for the decline in meadow bird populations in the last decades has been the more intensive harvest of grass on meadows in the Netherlands. However, rewetting of drained peatlands and implementing paludiculture would most probably imply even less space for meadow birds (personal note - Alex Schotman). Moreover, important meadow bird sites are often located in peat meadows, as is depicted on the map. However, note that other biodiversity such as protected sundew, aquatic warbler and orchids can be improved as a result of paludiculture.

\subsection{Social and cultural aspects}
% Cultural landscape
Cows grazing on Dutch meadows is something that many Dutch people familiarize themselves with. Moreover, according to a partitioner at a 'veenmiddag' in the province of Noord-Holland, 'the peat meadows are a special cultural landscape with an openness that has not altered over decades \citep{dosker}. In special, the partitioners wanted to keep the typical ditch structures as it has always been. These ditch structures are known as the "Slagenlandschap". Another attendent, someone who lived in a city mentioned that it is not necessary to maintain the landscap as it is, but more to look at opportunies such as energy landscapes close to home \citep{dosker}. All in all, the attendents agree that peat loss should be stopped and that paludiculture is a welcome alternative, at least in some places \citep{Dosker}

% Farmers openness

\section{spatial and temporal factors}

\subsection{Current system not ready}
The current Dutch use of peatlands is structured around dairy farming, for generations already. Farming of paludiculture would therefore be a major transition for the current land users. Based on participatory observation at a number of conferences, theme days and knowledge expeditions that were performed for this thesis, a few key issues were collected that reflect the current system's unability to start with the transition to paludiculture. 

First of all, paludiculture is not high on the political agenda, yet. Subsidies for paludiculture practices are not in play. Sometimes, regulations in play are opposing paludiculture practices. For example, in Germany, cattail may not be farmed since it is listed as a protected species. 

Secondly, Dutch farmers and agrologists are not convinced about making a transition to paludiculture. There is not enough financial security to make the transition to be a paludiculture farmer. However, paludiculture could be an addition to the current practices farmers perform. The nutrient runoff from their farmyard could be used for paludiculture crops, and paludiculture crops could be an addition to fodder for the cows. 

Thirdly, markets are currently underdeveloped. Paludiculture breeding to obtain the most efficient, the most robust plants is in its infancy. Since large scale production is not taking place, factories are not built and usage of the products is too expensive. 

\subsection{small steps are needed, looking for custom designed local solutions }
Although the current market, politics and farmers are not ready yet for large scale paludiculture implementation, small steps should be taken already to allow for future sustainable paludiculture practices. First of all, better sooner than later, the subsidy system for paludiculture should be updated so that farmers have more financial security. 

Secondly, each location is different, and therefore custom made solutions should be made, depending on the local interests at play. Whether these interests are protection of certain bird species and therefore taking a lower density of cattail products, maintaining the "Slagenlandschap" cultural look of the landscape by farming peat moss instead of cattail, nutrient filtering at farmyards or highly economic production where other social or ecological factors are of less importance.



%\chapter{Issues of Sustainability}

In this chapter, a number of sustainability issues will be addressed. Both bottlenecks and additional opportunities will be presented and discussed.

\section{Cultural and economic bottlenecks and opportunities}

\subsection{Farmers transition}
In the Netherlands, ... hectares of dairy farming takes place. In total, ... farmers are occupied to farm these lands. Since .. \% of the dairy farming surface is peat area (see Annex ..), we can estimate that approximately ... farmers farm the peat meadows in the Netherlands. Often, these farmers own the land and invested to build large sheds on their grounds. If paludiculture is to take place, these farmers have to make a transition to a different business model. \citet{sindram2017} interviewed a number of farmers to verify their openness towards cattail farming. The conclusion was that ... 

\subsection{available markets}
Paludiculture products is still in its infancy. A multitude of potential purposes is known, but a large scale market such as the dairy industry is not (yet) at play. A large problem that seems to exist is the expensive harvesting mechanism that has to  be performed. However, developments are taking place at the moment to make the harvesting procedure more cost-effective. Moreover, 


\subsection{political support}  
The farming industry in the Netherlands is heavily subsidized. However, paludiculture practices are not seen as agriculture yet, but as nature protection. Therefore, farmers cannot get subsidies for paludiculture. Moreover, as many people note, transition of a dairy farm to paludiculture is not easy and therefore, political support would help. At the moment, there are no plans to help farmers make the transition.

\subsection{cultural landscape}
The Netherlands is well known for its 'Slagenlandschap'. This is basically agricultural areas with many ditches. In Figure .., it can be seen where these areas can be found. If peat areas are gone, these areas can be used for other purposes than dairy farming, for example for potato or maize crops. The typical Dutch landscape would then disappear. However, maintaining the peat would imply to rewet the agricultural lands. Rewetting and applying paludiculture would alter the landscape but not necessarily the ditch structure for which the Netherlands is so well known. 

\section{nature conservation and other ecological issues}

\subsection{meadow bird areas}
A number of meadow birds species in the Netherlands is threatened. Bird protection organizations note that some species reached already a critical level of almost extinct. Research has been done showing areas that are important for meadow bird protection. In total, ... areas have been noted as important for meadow birds. In Figure .., it can be seen that some of these areas are also peat soils. In total, .. \% of important bird protection sites are peat areas. 

\subsection{nature sites protection}
In Figure .. nature sites and peat areas in the Netherlands are shown. Many nature sites are known to have problems with dryness. This often results in problems for biodiversity and the ecosystem. Paludiculture practices around these nature areas could prevent the water to seep away from the nature sites. On the map, in green, it is noted where paludiculture practices could go well together with nature area protection practices.

\subsection{water retention}
The Netherlands lies for a large part below sea level, see Figure .. To make the area more resilient against extreme weather events, it is noted to create more buffer retention zones. Paludiculture could be a means to keep productivity high and thereby also allowing the water to infiltrate every now and then. 

\subsection{water purification}
Paludiculture crops such as reed and cattail are commonly used in water treatment plants. It is known that these plants can easily take up loads of nutrients. If harvested, the materials can moreover be used for producing biogas or for other uses. Since eutrophication in the Netherlands is a rather large problem, paludiculture could help to solve this problem.



%
%
%\subsection{Peat oxidation in natural peat areas in the Netherlands}
%
%
%
%% How does it work? The practical of peat oxidation. 
%
%% Peat oxidation as a process
%
%% The unit of peat oxidation
%
%% Peat oxidation causes soil subsidence
%
%\subsubsection{Carbon balance in natural peat ecosystems}
%
%Natural peat ecosystems are characterized as swampy environments. Water levels generally are a few centimeters below ground level \citep{wichtmann2016paludiculture}
%
%% Is it relevant? In fen peatlands, water is provided by the groundwater, whereas in bog peatlands, water is provided by rain. 
%
%Natural peatlands are in general carbon sinks. More carbon is taken up than is emitted. Uptake of carbon takes place when plants photosynthesize carbon from the atmosphere into their stems, roots and leaves. When the plants die, some organic matter cannot decompose due to anaerobic conditions. With these layers of dead plant material, peat layers are slowly formed. Although peatlands are carbon sinks, some carbon is emitted in the form of methane (CH\sous{4}). That decomposition rates are slower than uptake rates, can be accounted for by 'slow rates of decomposition associated with cool temperatures, anoxic  conditions,  functionally  limited  decomposer  communities,  and  litter  and  organic  matter  substrates  that are naturally slow to decompose' \citep{moore2007litter}.
%
%In other words, if peatlands are in anaerobic conditions, they can be carbon sinks. However, when peatlands are drained and thus aerated, organic matter can decompose. That process of decomposition by micro organisms under aerobic conditions is called \textit{peat oxidation} \citep{erkens2016double}.
%
%\citet{smolders2013waterkwaliteit} note that peat oxidation is actually a redox reaction. The organic material functions as a energy giver (reductor) and if an oxidator is present, the organic material can be 'reduced'. Since oxygen is a very strong oxidator, if it is present in the peat, the organic matter will quickly be reduced. If other reductors are available, such as NO3 and CO2.
%
%% First story about natural systems that are sinks
%\subsubsection{Drainage in Dutch peatlands \& associated peat oxidation}
%
%In the Netherlands, we have a long history of drainage of peatlands. This chapter provides a short history recap of peatland usage and drainage.
%
%% Is it relevant? The Netherlands are what we could call a typical 'peat' country. Untill approximately 1200 years ago, more than 40\% of the Dutch surface consisted of peat \citep{vos2015origin, de2008vergeten}. These peat areas came up to several meters above sea level \citep{de2008vergeten}. Currently, only 7\% of the Dutch land surface consist of peat areas \citep{de2004verbreiding}. 
%
%Dutch peatlands have had various uses over time. Before 1500, peat areas were mainly exploited for arable farming \citep{ettema2005boeren}. After 1500, peat areas were found to wet for arable land use. Therefore, windmills were used to drain water away. Since the Dutch peat areas where too wet for crop farming, the main land use became livestock farming \citep{brouns2016effects}. Peat was furthermore used as a fuel source \citep{van1996turfwinning}.
%
%% Zeeuw; gouden eeuw door turf verklaart?
%
%Currently, most of the Dutch peatland areas in the Netherlands are drained with pump installations \citep{brouns2016effects}. As a result, Dutch peatlands have subsided upto a couple of meters \citep{de2008vergeten}.
%
%In the Netherlands, water levels are maintained by the 'waterschappen'. As a rule of thumb, these institutes maintain water levels in peat areas at -40 cm below the soil or lower. In the province of Friesland, water levels are sometimes maintained at -70 or lower. The constantly lowering of the water table has caused major losses of peat through peat oxidation. The average peat decline in the Netherlands was found to be .. cm \citep{erkens2016double}. \citet{erkens2016double} estimate that 19.8 km\sur{3} of peat have already been oxidized by drainage of Dutch peatlands. The authors note that the associated CO\sous{2} respiration is the same as a global CO\sous{2} concentration increase of \~0.39 ppmv. To put that into perspective, according to \citet{quadrelli2007energy} the change in CO\sous{2} concentration from pre-industrial situation to 2004 was 97 ppmv. Peat oxidation in the Netherlands thus can be accounted for by 0.004 \% of global human induced CO\sous{2} emissions.
% 
%The impact of ground water tables on subsidence and peat oxidation has been well studied. A linear correlation between water level and subsidence was found by \citet{van2010emission}. The researchers studied the effect of ditch water level and lowest ground water level on subsidence rates. They found the following linear relations:
%
%% relation 1: LGW and Subsidence in pure peat
%% relation 2: Ditch level and Subsidence in pure peat
%% relation 3: LGW and Subsidence in clayi peat
%% relation 4: Ditch level and Subsidence in clayi peat
%
%The results can be found in figure \ref{fig:oxrates}. 
%
%%In general, with a higher water table, less peat will oxidize. However, differences in soil affect the oxidation of peat. Peat with clay oxidizes less than pure peat. \citet{van2010emission} quantified the relationship between water table and CO\sous{2} emissions. They found a linear relationship between ditch water level and subsidence and lowest ground water level and subsidence. 
%
%
%
%According to a meta analysis by \citet{wichtmann2016paludiculture}, mean annual water level could explain the net GHG balance with a 95 \% confidence interval. Respectively, they found that at an annual mean water level of -50, and 0 cm, about 20 versus 8 ton CO\sous{2} equivalents ha\sur{-1} a\sur{-1} was emitted. % where was the GHG balance measured? in peat lands? What was the meta analysis about?
%
%All in all, it can be stated that the more peat is confronted with oxygen, the more CO\sous{2} is emitted. However, soil characteristics also play a role.
%
%% emissions of CO2
%% The authors note that the organic matter density in an average peat area is 103 kg m \sus{-3}.
%
%\subsubsection{Characteristics of soil affect peat oxidation}
%
%\citet{stouthamer2008toelichting} distinguished six reasons for vulnerability of organic matter of peat soils to oxidize: 
%
%\begin{enumerate}
%\item{Existence and thickness of mineral layer such as clay or sand}
%\item{Depth and thickness of peaty layers in the soil}
%\item{The organic content of the peaty layers}
%\item{The presence of mineral material below the peaty layer}
%\item{Depth of tillage}
%\item{Ground water tables}
%\end{enumerate}
%
%The last item (ground water table) was already discussed in the previous paragraph. This paragraph will deal with the first four items, which are all characteristics of the soil. 
%
%% add literature data
%
%\subsubsection{Peat oxidation causes subsidence}
%
%Next to greenhouse gas emissions, peat oxidation also causes soil subsidence. \citet{SCHOTHORST1977265} calculated that in the Netherlands, more than 50\% of the soil subsidence is caused by peat oxidation. The other 50\% is caused by compaction and consolidation of the soil.
%
%\subsection{Spatial distribution of oxidation potential of different peat types in the Netherlands}
%
%In 2008, \citet{stouthamer2008toelichting} mapped the oxidation potential of peat soils in the province of Utrecht. They distinguished six reasons for vulnerability of organic matter of peat soils to oxidize. 
%
%\begin{enumerate}
%\item{Existence and thickness of mineral layer such as clay or sand}
%\item{Depth and thickness of peaty layers in the soil}
%\item{The organic content of the peaty layers}
%\item{The presence of mineral material below the peaty layer}
%\item{Depth of the tillage}
%\item{Ground water tables}
%\end{enumerate}
%
%The authors note that the last two factors can be changed by mankind, but the other factors are fixed to the soil present. Based on the first four factors, the authors have made indices for each soil type that represent the oxidation potential of that peat type. In Utrecht, the researchers found 24 peat types. For each of these peat type, they calculated an index. These soil types and corresponding indices can be found in Annex ...
%
%In my thesis, I try to upgrade the work of \citet{stouthamer2008toelichting} to the rest of the Netherlands. In order to do so, I used a map with soil types made by Environmental Research Wageningen \citep{}. In their data set, 56 types of peat can be distinguished (Annex ..). For this thesis, indices for the 32 missing peat types were determined. To be able to do so, we must first understand how the indices are built up.
%
%\subsubsection{Building peat oxidation vulnerability indices}
%
%To calculate the indices \citet{stouthamer2008toelichting}, soil structures of the peat types were analysed and categorized for 3 factors, namely (1) soil type of the layers, (2) depth of the layers, (3) organic matter content in the layers and (4) profile type of the layers.
%
%\subsubsection{something about peat types}
%
%There are many different sorts of peat. In the Netherlands, 56 types of peat can be found. The classification is built up out of 2 components: (1) upper soil type and (2) lower soil type. The first component tells something about the abundance and types of mineral layers in the top soil and the second component tells something about the peat material type and mineral type of the layers below the top soil layer. An illustration can be found in Figure \ref{fig:soilclass} and an overview is given in Table \ref{tab:soilclass}.
%
%\begin{figure}
%    \centering
%    \includegraphics[scale=0.5]{figures/Soilclassificationdrawing} 
%    \caption{Soil classification}
%    \label{fig:soilclass}
%\end{figure}
%
%\begin{table}[htbp]
%\caption{Classification of peat soil}
%\begin{center}
%\begin{tabular}{|l|l|}
%\hline
%\textbf{Soil type} & \textbf{Letter} \\ \hline
%\multicolumn{ 2}{|c|}{Top layer} \\ \hline
%More than 10\% clay & h \\ \hline
%Less than 10\% clay & a \\ \hline
%(sabulous) clay & p \\ \hline
%Sand & z \\ \hline
%\multicolumn{ 2}{|c|}{Lower layer} \\ \hline
%Forest peat & b \\ \hline
%Peat moss peat & s \\ \hline
%(reed)sedge peat & c \\ \hline
%(sedge)reed peat & r \\ \hline
%(sabulous) clay & k \\ \hline
%sand without humuspodzol & z \\ \hline
%sand with humus podzol & p \\ \hline
%Unripened peat & o \\ \hline
%\end{tabular}
%\end{center}
%\label{tab:soilclass}
%\end{table}
%
%
%
%\subsubsection{Building peat oxidation index for 'Waardveengronden'}
%
%Since the peat type 'Waardveengronden' does not exist in Utrecht, no index was yet calculated. Therefore, I had to calculate the index from scratch. I did that in the same manner as the authors of the article did. The built up structure of 'Waardveengronden' can be seen in Table \ref{tab:waardveencalc}. Using the methodology of 'citet{stouthamer2008toelichting}, an index value for these soils could be found. An average value of 0.37 was found for 'Waardveengronden'. The procedure to calculate this index can be found in Appendix \ref{app:calcwaardveengrond}
%
%\begin{table}[htbp]
%\caption{Build up structure of Waardveengronden}
%\begin{center}
%\begin{tabular}{|l|l|}
%\hline
%\multicolumn{1}{|c|}{\textbf{Depth}} & \textbf{Description} \\ \hline
%0-4 cm & Dark Humus rich, heavy clay \\ \hline
%4-15 cm & Slow graduation towards lower layer \\ \hline
%15-30 cm & Dark, light humusrich, heavy clay \\ \hline
%30-40 cm & Black peat \\ \hline
%40-70 cm & Oxidated peatmoss peat \\ \hline
%70-120 cm & Reduced peatmosspeat \\ \hline
%\end{tabular}
%\end{center}
%\label{tab:waardveencalc}
%\end{table}
%
%\subsubsection{Interpolating indices}
%
%For a number of peat soil types, indices were not calculated. In order to do so, I used the method of standard deviations to interpolate current indices to other indices. I did this by (1) calculating average values for the main peat types after which (2) I could calculate standard deviations for the sub peat types. As a result, (3) I could add standard deviations of the sub peat types to main peat type averages for the overall peat types that were unknown. As a result, I could calculate indices for .. of the .. peat types. The result can be found in Appendix \ref{app:peatoxindices}.
%
%\subsubsection{Displaying peat oxidation vulnerability in the Netherlands}
%
%Now that I knew the indices of the peat soil types, I could combine the indices with a soil type map of the Netherlands. I decided to make four categories of peat oxidation potential, which are listed in Table \ref{tab:oxpotcategories}.
%
%\begin{table}[htbp]
%\caption{Categories of peat oxidation potential}
%\begin{center}
%\begin{tabular}{|l|r|}
%\hline
%\textbf{Index value range} & \multicolumn{1}{l|}{\textbf{Category)}} \\ \hline
%0.75 - 1.00 & Very High \\ \hline
%0.50 - 0.75 & High \\ \hline
%0.25 - 0.50 & Low \\ \hline
%0.00 - 0.25 & Very Low \\ \hline
%No Data & No Data \\ \hline
%\end{tabular}
%\end{center}
%\label{tab:oxpotcategories}
%\end{table}
%
%The produced maps of this exercise can be found in Figures \ref{fig:oxpotfriesland} and \ref{fig:oxpotwestnl}. Insight in the cumulative size of the areas can be found in Table \ref{tab:oxpotareacalcfriesland} and \ref{tab:oxpotareacalcwestnl}.
%
%\begin{table}[htbp]
%\caption{Size of four categories of peat oxidation potential in the province of Friesland, the Netherlands}
%\begin{center}
%\begin{tabular}{|l|r|}
%\hline
%\textbf{Oxidation Potential} & \multicolumn{1}{l|}{\textbf{Area (ha)}} \\ \hline
%Very High & 83461 \\ \hline
%High & 0 \\ \hline
%Low & 40517 \\ \hline
%Very Low & 24038 \\ \hline
%No Data & 2230 \\ \hline
%Total & 150246 \\ \hline
%\end{tabular}
%\end{center}
%\label{tab:oxpotareacalcfriesland}
%\end{table}
%
%
%\begin{figure}
%    \centering
%    \includegraphics[scale=0.5]{figures/oxpotentialfriesland} 
%    \caption{Oxidation potential of peat soils in Friesland. The darker the area, the higher the potential for peat oxidation.}
%    \label{fig:oxpotfriesland}
%\end{figure}
%
%\begin{figure}
%    \centering
%    \includegraphics[scale=0.5]{figures/oxpotentialwestnl} 
%    \caption{Oxidation potential of peat soils in the western part of the Netherlands. The darker the area, the higher the potential for peat oxidation.}
%    \label{fig:oxpotwestnl}
%\end{figure}
%
%\begin{table}[htbp]
%\caption{Size of four categories of peat oxidation potential in the western part of the Netherlands}
%\begin{center}
%\begin{tabular}{|l|r|}
%\hline
%\textbf{Oxidation Potential} & \multicolumn{1}{l|}{\textbf{Area (ha)}} \\ \hline
%Very High & 126410 \\ \hline
%High & 19701 \\ \hline
%Low & 33152 \\ \hline
%Very Low & 67590 \\ \hline
%No Data & 2317 \\ \hline
%Total & 249170 \\ \hline
%\end{tabular}
%\end{center}
%\label{tab:oxpotareacalcwestnl}
%\end{table}
%
%
%
%\subsection{Thickness of peat layers in the Netherlands}
%
%\
%
%\subsection{Water management change scenario's of paludiculture implementation}
%
%% possible way to calculate decrease in peat oxidation if water level in catchment area is heightened: Start with using the soil type data set to only keep peat soils in the raster. Then, Start with determining subsidence rates for the resting raster cells based on current water levels. That can be done by substracting the height with the water level and then using a formula that relates water level with subsidence. we then have a raster with subsidence rates of mm per year. Then if the rates are known, we can set the water table higher. We can again calculate subsidence rates for the peat areas in the catchment. Then we have to raster files with subsidence rates. we substract the raster files and obtain difference in subsidence rates. Following, based on the size of a raster cell, we can calculate the loss of peat per cubic meter per raster cell by multiplying it by the subsidence difference in the cell. At last, we sum up all the raster cell values and we get the decrease in peat oxidation per year.  
%
%It is known that wet crops can be applied in different spatial scales. This has to do with the structure of catchment areas that is common in the Netherlands. Water levels are managed in the Netherlands with larger catchment areas (peilvak "PV") and smaller catchment areas within the 'peilvak' (onderbemalingen "OB" and hoogwatervoorzieningen "HV"). The water level over a catchment area (except for the onderbemaling and the hoogwatervoorziening) is the same, but height difference can exist within a catchment area. Therefore, a first option is to implement wet crops in lower lying areas, since it is easier to flood these areas and moreover, the lower areas are more vulnerable for oxidation. A second option is to use current 'onderbemalingen' for paludiculture crops. .. note that in these areas, it is easier to give input water flows. A third option would be to maintain water levels on higher levels then the surrounding, in 'hoogwatervoorzieningen'. The fourth option would be to flood whole 'peilvak' areas. 
%
%A first step will be to give an indication of carbon balance changes in either these four options. In fact, option 2 to 4 will have easy calculations since the whole area of paludiculture will change its carbon balance equally. However, option 1, using lower areas of the 'peilvak' needs some extra consideration. That is because raising the water table towards the lower areas, will influence peat oxidation in the higher parts as well. In order to give a good estimation of carbon balance changes, it is first good to check average height differences in 'peilvak' areas. A separate section of this chapter will give insight into height differences by taking ten 'peilvak' areas and check for height differences.
%
%Subsequently, a scenario study will give an indication of carbon balance changes after rewetting a 'peilvak' partly. Afterwards, in a final section, insight will be provided about the actual contribution of peat oxidation by 'onderbemalingen', 'hoogwatervoorzieningen' and 'peilvakken'. By extrapolating the found changes in carbon balances, a national indication for carbon changes can be given for each of the (combined) options. 
%
%\subsubsection{carbon balance changes options 2 - 4}
%
%\subsubsection{average height differences in 'peilvak' areas}
%
%
%\begin{figure}
%    \centering
%    \begin{subfigure}[a]{0.3\textwidth}
%    	\includegraphics[scale=0.2]{figures/peilvak1} 
%    	\caption{Height differences in catchment 1}
%    	\label{fig:peilvak1}
%    \end{subfigure}	
%    \quad
%    \begin{subfigure}[a]{0.3\textwidth}
%    	\includegraphics[scale=0.2]{figures/peilvak2} 
%    	\caption{Height differences in catchment 2}
%    	\label{fig:peilvak2}
%    \end{subfigure}	
%    \caption{Height differences in catchment areas}
%    \label{fig:azolla}
%\end{figure}
%
%
%\subsubsection{carbon balance after rewetting 'peilvak' partly by doing scenario study}
%
%\subsubsection{national indication for carbon balance changes as a result of implementing either of the four options}
%
%
%% mogelijkheden om hele peilvakken te gebruiken, of delen van peilvakken of onderbemalingen. Dan zou je de laagste delen van peilvakken kunnen gebruiken. Door de het water op te hogen tot het laagste deel, zal ook het hoogste deel minder drooggelegd zijn. Dat zou een indirect bij-effect kunnen opleveren voor bodembehoud. Bestaat dit indirect bij-effect? Om dat te bepalen, moeten we eerst weten of het fysisch mogelijk is een deel van een peilvak te gebruiken. Dat kan bepaald worden door te kijken of er een substantieel hoogteverschil is binnen het peilvak. Maar wat is dan substantieel? Daarvoor bepaal ik dat het laagste kwartiel min het hoogste kwartiel meer dan 40 cm moet zijn.  
%
%% 3-7 case studies toetsen waarin hoogteverschil van peilvak met gelijk peil wordt bepaald. Hoogwaterniveaus worden uitgezonderd.
%
%\subsubsection{Effect of elevation differences in peilvakken on peat conservation when paludiculture is implemented} 
%
%% scenarios met deel van peilvak als paludicultuur en effect daarvan op indirecte veenbehoud.
%
%
%
%%Regarding the possibility of an area to be suitable for paludiculture, a first step would be to define what makes an area a potential site. In fact, all areas that can be flooded with water are suitable in first instance. In general, these are areas below sea level. Doing that, we obtain the following map of the Netherlands (Figure \ref{fig:maplow}):
%%
%%\begin{figure}
%%    \centering
%%    \includegraphics[scale=0.5]{heightmap.png} 
%%    \caption{Areas in the Netherlands below sea level}
%%    \label{fig:maplow}
%%\end{figure}
%
%\subsection{Spatial distribution of areas surrounding nature areas in the Netherlands}
%
%\section{Sustainability of paludiculture}
%
%% Water management in The Netherlands is subdivided by 22 "Waterschappen" who all manage a certain piece of the Dutch water system. In the lower part of the Netherlands, water levels are maintained by pumping water away from ditches to reservoirs (boezems). The pumping is done per catchment area, a so-called "peilvak". The Netherlands has more than 10.000 of these catchment areas \citep{nijhuis2013polderkaart}. Sizes of the catchment areas vary from around 100 ha to 10000 ha \citep{Vermaat2010}. 
%
%%Catchment areas are sometimes shared with more farmers, so raising the water table might be problematic if not all farmers start with paludiculture. 
%
%\subsection{Ecological bottlenecks and co-benefits}
%
%\subsubsection{Meadow birds}
%
%Meadow bird populations in the Netherlands are threatened \citep{}. 
%
%The most important boundary conditions for meadow birds are openness of the landscape, a certain water table and moment of grass mowing \citep{teunissen2012op}. 
%
%% what is minimum water table?
%
%\subsubsection{Nature conservation sites}
%
%Various people have mentioned that paludiculture is very useful to implement around nature areas. Many nature areas are currently threatened by droughts because surrounding areas are declining. Paludicultures can deal better with higher water levels than current agriculture practices and thus can prevent water to seep away from nature areas to surrounding areas.
%
%\subsubsection{Water Storage}
%
%All low peat areas in The Netherlands are below sea level and declining further. With climate change, it is predicted that the sea level will rise and extreme weather patterns such as heavy rains will occur more frequently (). Therefore, chances on floods will increase in the future. In order to adapt to these problems, water storage places are important. Currently, it is sometimes hard for water board to decide what to do with excess water. Some paludiculture crops can handle temporary high water levels well \citep{wichtmann2016paludiculture}. Cattail will function fine when the water level is raised with 50 cm for 2 or 3 weeks (). Other species such as peat moss are more vulnerable to fluctuating water levels. 
%
%
%
%\subsubsection{New sites for threatened species}
%
%\subsubsection{Nutrient mining}
%
%Dutch water systems are highly polluted with especially phosphorus \citep{}, but also with nitrogen. As explained in sections .. and .., both Azolla and Cattail species are known to be able to thrive well under high phosphorus and nitrogen contents. When harvested, these plants can function as nutrient mining species and thereby clean the water from pollution. 
%
%Moreover, raising the water level when implementing paludiculture, will decrease the mineralization of the peat. In turn, mineralization leads to extra nutrient loads since organic matter is broken down and N and P can leach. Moreover, paludiculture doesn't need input from fertilizer, thus reducing nutrient pollution \citep{wichtmann2016paludiculture}.
%
%\subsubsection{Alternative for fossil fuels}
%
%
%\subsection{Economic bottlenecks}
%
%\subsubsection{Profits}
%
%The major part of the Dutch peat lands is cultivated with grass. Currently, … \% of the agriculture on peatlands are grasslands. The grass is grown solely to feed cattle, mainly for cows but also for sheep, goats and horses. The majority of the cows - .. \% - live in stables and go out in the summer months to graze. In general, grasslands are mown 5 times a year. An average Dutch cow produces 10.000 liters of milk per year (). These practices can be called intensive farming since production is put as a priority. Only 20 \% of milk production is used for consumption in the Netherlands. The rest is exported to other countries, both as milk as well as cheese. The dairy industry gives … jobs in the Netherlands. In 2015, dairy exports were 7.5 billion euro’s, more than the 6.3 billion euro’s that vegetables gave ()
%
%Various studies are performed that compare the production of paludiculture crops with the dairy industry. Some studies have found that paludiculture produce are currently not competitive with  dairy farming. For example, … and … evaluated whether Common Reed can compete economically with dairy farming. They found that Common Reed is currently less profitable than dairy farming. A study by the …, calculated the expected profits of growing big cattail instead of doing dairy farming. In this study, the authors used a scenario with carbon credits and one scenario without carbon credits. The study showed that cattail farming can give more profits than dairy farming. Still, various people argue that markets for paludiculture products are still small or yet in development. For example, common reed is a product widely used in the Netherlands for thatching. However, only a small proportion ( \%) is produced in The Netherlands. Other countries simply produce reed cheaper. Azolla is mentioned as a promising paludiculture since it has a high production rate and contains a lot of proteins (). However, there is currently no company that uses Azolla. Furthermore, the question is how much paludiculture production the market can handle. At the moment, it is known that many different products can be made from cattail. Therefore, many companies would like to get raw cattail. However, production is almost zero in The Netherlands. If the Dutch landscape can produce tons of cattail, the question is whether the market is still in need of these products. 
%
%\subsubsection{Investments}
%
%Many farmers have invested largely in new stables, equipment or other instruments that help them to do dairy farming. The divestment often lasts at least 25 years, which means that is not easy for a farmer to give up his current practices and start with a new business which also needs large investments. Paludicultures differ for the amount of start capital and investments that are needed. Cattail farming costs .. money to start with and will be won back after … years ().
%
%\subsubsection{Water storage}
%
%\subsubsection{Infrastructure damage}
%
%\subsubsection{Pumping costs and extra water storage costs}
%
%Since paludiculture crops require water levels close to the surface, water levels have to be maintained at such a level. However, in summer months, evaporation occurs, affecting water levels to drop. In natural healthy water systems, this would not be a problem because water is provided enough. However, in current water systems, water needs to be imported in summer months. To prevent water level to drop below -20 cm, a mean annual irrigation of 110 m\sur{3} ha\sur{-1} is needed.  \citep{wichtmann2016paludiculture}. 
%
%\subsection{Social-Cultural bottlenecks}
%
%\subsubsection{Transition to other land use}
%
%The typical Dutch landscape that is known worldwide is a peat meadow inclined by ditches with cows grazing. … Estimate that this image is worth money as an export product. The authors suggest that the typical Dutch landscape is worth ..\% of Dutch GDP. Moreover, many Dutch people find it very important to keep the landscape. The peat meadows seem to be part of people’s national identity. Currently, many action groups are rallying for protection of the peat meadows. An example is .. Moreover, recently, the Dutch ‘verkavelingspatroon’ is proposed as a UNESCO heritage site. 
%
%Implementation of paludiculture crops would mean a change of appearance of the Dutch landscape. In an interview with .., .. explains that it Dutch people do not want to see mono cultures of paludicultures. This is supported by research by .. in which it was found that .. However, many paludicultures do not necessarily completely change the Dutch landscape appearance. For example, common reed is already very common in Dutch landscape. Cattail is a crop that has similarities in appearance with growing maize. A field of azolla or peat moss is green and looks like grassland. In most cases, ditch structures can be kept, so that the cultural heritage of .. will be kept intact. 
%
%Currently, most farmers in peat areas farm cows for dairy products. The farmers often do this already for generations. Most dairy farmers do not see a future where they will farm paludiculture crops instead (). However, in the recent past (50 years ago) it was common for farmers to have a small field of e.g. cattail in the lower areas that could be fed to the cows. That inclines that the idea of combining dairy farming with paludiculture is not so far-fetched. 
%
%Most Dutch areas are currently owned by dairy farmers. However, there was a trend of nature organizations (Staatsbosbeheer and Natuurmonumenten) to buy up land in peat areas. Currently, in the western peat areas, 13000 hectares are owned by nature organizations (Janssen, 2009). Recently, Staatsbosbeheer has mentioned that it will not further decrease the water level in a certain land. That means that the farmers who now lease the land can still do that later or decide not to. It means that current dairy farming practices are not possible anymore, unless they use new technologies (see 1.1.4) or implement other practices such as paludicultures. 
%
%\subsubsection{Maintaining the typical Dutch 'Slagenlandschap'}
%
%To maintain the characteristic landscape of the Dutch peat meadows inclined with ditches, peat subsidence should be stopped. The only way to do so is either nature development or paludiculture \citep{Dosker2017}. The author notes that maintaining high water tables while keeping dairy farming practices will only slow down the peat decline. 
%
%\subsubsection{Archeology conservation}

\clearpage

\begin{appendices}

%\addappheadtotoc
\appendixpage

\chapter{Calculation index value Waardveengronden}
\label{app:calcwaardveengrond}

In Table \ref{tab:indexcalcwaardveengronden} the method to get the index for Waardveengronden is showed. 

\begin{table}[ht]
\small
\caption{Calculation of index value for Waardveengronden}
\label{tab:indexcalcwaardveengronden}
\begin{tabulary}{1.0\textwidth}{|C|L|L|L|L|}
\hline
Diepte (cm -mv) & Textuur-/organische- stofklasse  & Indexwaarde factor b (per 10 cm) & Indexwaarde factor c (per 10 cm) & Vermenigvuldiging Indexwaarde factoren b en c (per 10 cm)  \\ \hline \hline
10 & matig zware klei & 0.9 & 0 & 0 \\ \hline
20 & zeer zware klei & 0.8 & 0 & 0 \\ \hline
30 & zeer zware klei & 0.7 & 0 & 0 \\ \hline
40 & zwart spalterveen & 0.6 & 0.7 & 0.42 \\ \hline
50 & veenmosveen & 0.5 & 0.7 & 0.35 \\ \hline
60 & veenmosveen & 0.4 & 0.7 & 0.28 \\ \hline
70 & veenmosveen & 0.3 & 0.7 & 0.21 \\ \hline
80 & veenmosveen & 0.2 & 0.7 & 0.14 \\ \hline
90 & veenmosveen & 0 & 0.7 & 0 \\ \hline
100 & veenmosveen & 0 & 0.7 & 0 \\ \hline
110 & veenmosveen & 0 & 0.7 & 0 \\ \hline
120 & veenmosveen & 0 & 0.7 & 0 \\ \hline
 & & & Sum & 1.4 \\ \hline
 & & & Factor for profile & 0.6 \\ \hline
 & & & & 0.84 \\ \hline
 & & & Final Index & 0.37 \\ \hline
\end{tabulary}
\end{table}

\clearpage

\chapter{\bf Peat oxidation indices}
\label{app:peatoxindices}

\begin{center}
\begin{longtable}{|l| p{11cm} | l |}
	\caption{Peat oxidation index values for all peat types that can be found in the Netherlands.} \label{tab:indexpercode} \\

\endfirsthead

\multicolumn{3}{c}%
{{\bfseries \tablename\ \thetable{} -- continued from previous page}} \\
\hline \multicolumn{1}{|c|}{\textbf{Code}} &
\multicolumn{1.0}{c|}{\textbf{Description}} &
\multicolumn{1.0}{c|}{\textbf{Index}} \\ \hline 
\endhead

\hline \multicolumn{3}{|r|}{{Continued on next page}} \\ \hline
\endfoot

\hline \hline
\endlastfoot
\hline
\textbf{Code} & \textbf{Description} & \multicolumn{1}{l|}{\textbf{Index}} \\ \hline


Vo & Vlietveengronden  & 1 \\ \hline
Vr & Vlierveengronden op rietveen of zeggerietveen  & 1 \\ \hline
hVr & Koopveengronden op rietveen of zeggerietveen  & 0.98 \\ \hline
aVs & Madeveengronden op veenmosveen  & 0.94 \\ \hline
hVs & Koopveengronden op veenmosveen  & 0.93 \\ \hline
aVc & Madeveengronden op zeggeveen, rietzeggeveen of broekveen  & 0.93 \\ \hline
Vs & Vlierveengronden op veenmosveen  & 0.92 \\ \hline
Vc & Vlierveengronden op zeggeveen, rietzeggeveen of (mesotroof) broekveen  & 0.89 \\ \hline
Vb & Vlierveengronden op bosveen (of eutroof broekveen)  & 0.87 \\ \hline
hVb & Koopveengronden op bosveen (of eutroofbroekveen)  & 0.86 \\ \hline
hVc & Koopveengronden op zeggeveen, rietzeggeveen of (mesotroof) broekveen  & 0.86 \\ \hline
Vz & Vlierveengronden op zand zonder humuspodzol, beginnend ondieper dan 120 cm  & 0.83 \\ \hline
hVd & Koopveengronden op bagger, verslagen veen, gyttja of andere veensoorten  & 0.81 \\ \hline
hVz & Koopveengronden op zand, beginnend ondieper dan 120 cm  & 0.81 \\ \hline
aVz & Madeveengronden op zand zonder humuspodzol, beginnend ondieper dan 120 cm  & 0.81 \\ \hline
aVp & Madeveengronden op zand met humuspodzol, beginnend ondieper dan 120 cm  & 0.81 \\ \hline
Vd & Vlierveengronden op bagger, verslagen veen, gyttja of andere veensoorten & 0.81 \\ \hline
Vp & Vlierveengronden op zand met humuspodzol, beginnend ondieper dan 120 cm  & 0.78 \\ \hline
Vk & Vlierveengronden op (meestal niet-gerijpte) zavel of klei, beginnend ondieper dan 120 cm  & 0.68 \\ \hline
hVk & Koopveengronden op (meestal niet-gerijpte) zavel of klei, beginnend ondieper dan 120 cm  & 0.6 \\ \hline
kVr & Waardveengronden op rietveen of zeggerietveen  & 0.53 \\ \hline
kVs & Waardveengronden op veenmosveen  & 0.46 \\ \hline
kVc & Waardveengronden op zeggeveen, rietzeggeveen of (mesotroof) broekveen  & 0.43 \\ \hline
kVb & Waardveengronden op bosveen (of eutroof broekveen) & 0.41 \\ \hline
kVd & Waardveengronden op bagger, verslagen veen, gyttja of andere veensoorten  & 0.37 \\ \hline
kVz & Waardveengronden op zand, beginnend ondieper dan 120 cm  & 0.34 \\ \hline
Wg & Moerige eerdgronden met een moerige bovengrond of  moerige tussenlaag op gerijpte zavel of klei  & 0.32 \\ \hline
pVr & Weideveengronden op rietveen of zeggerietveen  & 0.3 \\ \hline
vWp & en een moerige tussenlaag Moerige podzolgronden met een moerige bovengrond  & 0.28 \\ \hline
vWz & Moerige eerdgronden met een moerige bovengrond op zand  & 0.28 \\ \hline
zVs & Meerveengronden op veenmosveen  & 0.24 \\ \hline
kVk & Waardveengronden op (meestal niet-gerijpte) zavel of klei, beginnend ondieper dan 120 cm  & 0.22 \\ \hline
zVc & Meerveengronden op zeggeveen. rietzeggeveen of broekveen  & 0.22 \\ \hline
pVs & Weideveengronden op veenmosveen  & 0.2 \\ \hline
pVc & Weideveengronden op zeggeveen, rietzeggeveen of (mesotroof) broekveen  & 0.18 \\ \hline
pVb & Weideveengronden op bosveen (of eutroof broekveen)  & 0.16 \\ \hline
pVd & Weideveengronden op bagger, verslagen veen, gyttja of andere veensoorten & 0.13 \\ \hline
zVz & Meerveengronden op zand zonder humuspodzol, beginnend ondieper dan 120 cm & 0.11 \\ \hline
zVp & Meerveengronden op zand met humuspodzol, beginnend ondieper dan 120 cm & 0.09 \\ \hline
pVz & Weideveengronden op zand, beginnend ondieper dan 120 cm  & 0.07 \\ \hline
pVk & Weideveengronden op (meestal niet-gerijpte) zavel of klei, beginnend ondieper dan 120 cm  & 0.06 \\ \hline
zWp & Moerige podzolgronden met een humushoudend zanddek en een moerige tussenlaag  & 0.06 \\ \hline
zWz & Moerige eerdgronden met een zanddek en een moerige tussenlaag op zand  & 0.06 \\ \hline
kWp & Moerige podzolgronden met een zavel- of een kleidek  & 0.05 \\ \hline
kWz & Moerige eerdgronden met een zavel- of kleidek en een moerige tussenlaag op zand  & 0.05 \\ \hline
hEV & Aarveengronden  & \multicolumn{1}{l|}{No Data} \\ \hline
iVs & Veengronden met een veenkoloniaal dek op veenmosveen & \multicolumn{1}{l|}{No Data} \\ \hline
iVc & Veengronden met een veenkoloniaal dek op zeggeveen, rietzeggeveen of moerasbosveen  & \multicolumn{1}{l|}{No Data} \\ \hline
iVz & Veengronden met een veenkoloniaal dek op zand zonder humuspodzol, beginnend ondieper dan 120 cm  & \multicolumn{1}{l|}{No Data} \\ \hline
iVp & Veengronden met een veenkoloniaal dek op zand met humus podzol, beginnend ondieper dan 120 cm  & \multicolumn{1}{l|}{No Data} \\ \hline
iWp & Moerige podzolgronden met een veenkoloniaal dek en een moerige tussenlaag  & \multicolumn{1}{l|}{No Data} \\ \hline
uWz & Moerige eerdgronden met een mineraal dek 5-8\% lutum en een moerige tussenlaag op zand  & \multicolumn{1}{l|}{No Data} \\ \hline
iWz & Moerige eerdgronden met een veenkoloniaal dek en een moerige tussenlaag op zand  & \multicolumn{1}{l|}{No Data} \\ \hline
AVk & Veenafbraakgebied & \multicolumn{1}{l|}{No Data} \\ \hline
AVo & Veen in ontginning & \multicolumn{1}{l|}{No Data} \\ \hline
\end{longtable}
\end{center}


\end{appendices}

\bibliography{/home/merijn/Documents/Wageningen/Thesis/other/tex/thesisbibliography}

\end{document}

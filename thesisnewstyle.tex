\chapter{Introduction}
%K Try to write a bit more towards the problem statement.
%K The paragraph of usage in the world and the Netherlands could be together; also the paragraphs of GHG or world and NL
%K More explanation of subsidation 

\section{Problem background and problem statement}

Natural peatlands are wet ecosystems. In natural peatlands, carbon is taken up by plants through photosynthesis of \ac{CO2} and stored in layers of dead vegetation \citep{clymo1984limits}.. Some carbon is emitted through anaerobic decomposition by micro-organisms as Methane (CH4), a strong \ac{GHG}. Overall, natural peat ecosystems are net carbon sinks because they store more carbon than they emit.

All over the world, peatlands are drained because of agricultural, forestry or other purposes. In the Netherlands, most peat areas in the Western part of the country are drained to allow dairy farming practices and in the North-Eastern part of the country to farm agricultural crops. As peatlands are drained, the peat is aerated, allowing the organic plant material to decompose through the process of oxidation. After aeration of peat, CH4 emissions are stopped but CO2 emissions increase. As a result, peatlands become net carbon sources. Although 1.5 \% of the global terrestrial surface are peat areas, peatlands account for 6\% of all global \ac{GHG} emissions \citep{joosten2012peatlands}. In the Netherlands, peat oxidation cause 2.5 \% of the total national GHG emissions \citep{van2010emission}. 

For the Netherlands to be able to achieve the Paris Agreement commitments, GHG emissions should be decreased with .. \%. Reducing GHG emissions caused by peat decomposition could be part of the measures to reduce national GHG emissions. However, returning to natural wet peatlands that are carbon sinks, would result in loss of current agricultural practices. Maintaining agriculture production while reducing GHG emissions caused by peat decomposition, wet agriculture is a solution. Wet agriculture while protecting organic soils is called paludiculture \citep{joosten2002wise}. Various authors have suggested that only with rewetting peatlands, peat decline can be fully stopped \citep{van2016dalende, wichtmann2016paludiculture}. Paludiculture is the only productive alternative after rewetting on drained soils \citep{wichtmann2016paludiculture}. It is therefore not suprising that awareness among various scholars and policy makers is rising that paludiculture will be a significant part of the future land use on Dutch peatlands \citep{abel2013database, wichtmann2016paludiculture, Wichmann20151063}. 

Paludiculture is a newly developing concept and interest in implementing paludiculture is growing. However, there is no overview of the actual places in the Netherlands where paludiculture is a sustainable alternative to reduce GHG emissions from peat soils. Therefore, this study will deal with the problem that:
\\
\textbf{It is currently unknown where paludiculture is a sustainable option to reduce GHG emissions caused by peat decomposition in the Netherlands}
\\
In the context of this study, sustainability is defined as reducing GHG emissions from peat decomposition while minimizing production loss, reduced nature protection and diminishing cultural values. 

\section{Goal and research questions}
The goal of this study is to give insight in the (spatial) opportunities of paludiculture as a sustainable option to minimize GHG emissions from Dutch rural peat areas. In order to reach the goal, the following main research question and sub research questions were defined:

\textbf{What are (spatial) opportunities of paludiculture as a sustainable option to reduce GHG emissions caused by peat decomposition in the Netherlands?

\begin{enumerate}
\item \label{rq:abiot} {What are market potentials and abiotic requirements of three paludiculture crops in the Netherlands?} 
\item \label{rq:GHG} {How can GHG emissions caused by peat decomposition in the Netherlands be reduced by paludiculture crops?}
\item \label{rq:sust} {Which sustainability factors need to be accounted for when implementing paludiculture in the Netherlands?}
\item \label{rq:spatial} {What are spatial opportunities of paludiculture as a sustainable option to reduce GHG emissions caused by peat decomposition in the Netherlands?}
\end{enumerate}

%K No Titles; chapter met capital C; first conclusion, then discussion
%In chapter \ref{ch:datamethods}: \nameref{ch:datamethods} the study site, spatial data collection and analysis other means of data collection and methodology are described. Subsequently, chapter \ref{ch:litpalghg}: \nameref{ch:litpalghg} will provide a literature study about potential paludiculture crops and experimental studies regarding reduction of GHG emissions from rewetted peat soils. In the \ref{ch:abiot}\sur{th} chapter: \nameref{ch:abiot} insight will be given in abiotic requirements of three paludiculture and to what extent some of these requirements are met by Dutch conditions. Hereafter, chapter \ref{ch:optloc}: "\nameref{ch:optloc}" deals with spatial analyses about peat mineralization potential of different Dutch peat soils, what catchments contain most peat and whether partially rewetting catchments is helpful to reduce GHG emissions from peat soils. Furthermore, chapter \ref{ch:sust}: "\ref{ch:sust}" will shed light on ecological, economic, social, cultural as well as spatial and temporal aspects of paludiculture. At last, \ref{ch:condis}: \nameref{ch:condis} provides insight in the conclusions that can be derived from this study and puts the study in a broader perspective.

\chapter{Promising paludiculture crops} \label{ch:abiotmarket}

This chapter addresses research question \ref{rq:abiot}. As a first step, the concept of paludiculture will be explained and the market potential of various promising species will be described. Then, a literature study will be presented about the abiotic requirements of three promising paludiculture species.

\section{The concept of paludiculture}

%Paludiculture definition
Paludiculture stems from the Latin word Palus, which means swamp. The term paludiculture was coined in 2007 and basically describes the productive usage of wetlands while reducing greenhouse gas emissions from peat areas \citep{wichtmann2007paludiculture}. 

%Number of suitable plant species for paludiculture in the Netherlands
A typical Dutch paludiculture crop is Common Reed, which is a water plant species used for thatching roofs \citep{wichtmann2016paludiculture}. Only recently, \citet{abel2013database} selected a list of 812 water plants that are globally suitable for cultivation. In their research, 184 plants were found suitable for Western Pomaria in Germany. Since the climate in this region is similar to the Dutch climate, 184 plants can be classified suitable to cultivate in the Netherlands. 
%Ask Susanne to provide a list

% Current paludiculture practices
Currently, some paludiculture practices already exist in the Netherlands. At least 8 million kilo Common Reed is harvested every year in the Netherlands. Usage is even higher, with at least 20 million kilo a year \citep{wichtmann2016paludiculture}. Also, the island of Terschelling is known to have cranberry plantations. Cattail and peat moss are two other common species in Dutch ecosystems that have been put forward as very promising species to cultivate \citep{abel2013database, van2013werk}. Also, Azolla, which is not native to the Netherlands, is mentioned as a promising species for paludiculture \citep{abel2013database, van2013werk}. 
% Maybe add the description of species in boxes or annex?
These three species will be elaborated on below.

\section{Market potential of three promising paludiculture crops in the Netherlands}

\subsection{Cattail}

% Determine whether to use latin names or English names.

%Difference between latifolia and angustifolia
\textit{Typha} or cattail is a plant genus endemic in Europe (see Figure \ref{fig:cattail}). Two prominent species of this genus are \textit{Typha latifolia} or broadleaf cattail and \textit{Typha angustifolia} or narrowleaf cattail. Both species were studied as a potential crop in prior agricultural plots in Germany in 1998 \citep{wild2001cultivation}. Both species were planted and its sprouts multiplied with a factor 7-10 in two months. The cattail family is known as a robust plant which can easily tolerate flooding and high nutrient contents \citep{wild2001cultivation}. \citet{heinz2011population} compared both cattail species and found a number of major differences between the species. First of all, broadleaf cattail produces ten times more offspring from shoots in optimal conditions suggesting it to be a very pioneering plant. Narrowleaf cattail is more adapted to a competitive strategy, the species has for instance a higher salt-tolerance \citep{heinz2011population}. A study by \citet{grace1982niche} showed that narrowleaf cattail can grow better in deeper water than common cattail due to its greater leaf surface area.

%Experiments 
Experiments with cattail are taking place in Germany since the '00s, and have recently started in the Netherlands as well. In Zegveld, a pilot of 0.5 hectares with broadleaf cattail started in 2012. In 'Het B\^utefjild', another pilot project was started only recently. 

%Usage of cattail
Cattail species have a history of usage everywhere around the world \citep{morton1975cattails}. \citet{morton1975cattails} notes that cattail in the Europe was used to stuff chairs and make winter protection for crops. Although cattail is hardly used nowadays, interest in cattail as a paludiculture crop is growing. Various authors have suggested cattail as a promising species \citep{morton1975cattails, heinz2011population}. They mention cattail promising, not only for its robustness, but also for its potential uses. Cattail has many purposes, which are listed in Table \ref{tab:typha}. The most promising usage is as isolation material. Due to its fire-resistant properties, it is a fire-proof alternative for isolation. It can be made into isolation boards or as a wall-filler. 

\subsubsection{Peat moss}

% peat moss in the Netherlands
Another plant that is put forward as promising is \textit{Sphagnum} or peat moss (see Figure \ref{fig:peatmoss}). Peat moss is a native species to the Netherlands that once dominated peat bogs in the Netherlands. The water supply in its natural condition comes solely from rain water. Experiments with peat moss started already in the 2001 in Germany \citep{gaudig2014sphagnum}. In 2013, a pilot project was started in 'Het Ilperveld', near Amsterdam \citep{van2013werk}. 

% alternative to horticulture substrate
Cultivation of peat moss is interesting since it is seen as a sustainable alternative to harvested peat for horticulture substrates. Currently, the best substrate for horticulture industry is still peat harvested from bogs. Every year, around 20 million m\textsuperscript{3} of peat is harvested in Europe to supply peat for horticulture \citep{altmann2008socio}. The highest quality is so-called white peat which has formed from sphagnum mosses over the last 3000 years \citep{gaudig2014sphagnum}. Peat moss farming is a good alternative to obtain substrates for horticulture. Using biomass from peat moss farming in horticulture works equally well or better than standard peat-based media. That is not surprising, since both alternatives have similar physiological qualities \citep{gaudig2014sphagnum}. 

% promosing peat moss species
According to \citet{gaudig2014sphagnum} and \citet{wichtmann2016paludiculture}, \textit{Sphagnum palustre} is one of the most promising species of the peat moss genus since it is highly productive. The authors furthermore mention that peat moss farming is already economically profitable. That is mostly because selling peat moss in niche markets results in high revenues. However, peat moss farming cannot yet compete with cheaply extracted white peat \citep{gaudig2014sphagnum}.

\subsection{Azolla}

% Azolla introduction
Another species that is put forward as interesting is \textit{Azolla}, a fast growing surface water plant very rich in proteins. The plant forms a symbiosis with \textit{Anabaena azollae}, a bacteria that fixes nitrogen from the air \citep{wagner1997azolla}. \textit{Azolla} is therefore never limited by nitrogen. \textit{Azolla} is not native in The Netherlands, but currently widely common in Dutch waters. \textit{Azolla} can be harvested for various purposes. Since a long time, \textit{Azolla} is used in Asian countries as green manure in rice paddies \citep{wagner1997azolla}. A promising use for the Netherlands is therefore also to use it as fertilizer. Potential purposes are mentioned by \citet{wagner1997azolla} and listed in Table \ref{tab:azolla}.

\section{Abiotic requirements of three promising paludiculture crops in the Netherlands}

\subsection{Cattail}

 %\citet{morton1975cattails} notes that seedlings of Common Cattail will die out if the water level is maintained at 45 cm from spring to midsummer. 

% water level
Cattail species are robust plants, but have some requirements for growing. Growth of mature plants of broadleaf cattail will be severely impaired by water depths of 63.5 cm \citep{morton1975cattails}. Narrowleaf cattail can survive in water depths up to 1.2 meters \citep{morton1975cattails}. \citet{dubbe1988production} also note that Cattail species can deal with water depths upto 50 and 115 cm above the ground. However, another study showed that the optimal water depth of broadleaf cattail is 22 cm above the ground \citep{grace1989effects}.
%What about minimum water levels?

% nutrients
\citet{ciria2005Typha} studied the potential of common cattail in waste water treatment. The authors found that common cattail did not have a problem with high concentrations of nitrogen (52 mg per liter) and phosphorous (23 mg per liter). Also, \citet{newman1996effects} found that adding 0.05 millilgram per liter P or 0.1 milligram per liter N was fine for broadleaf cattail growth.
% Provide a study that tells that higher N presence leads to more biomass production. If there is not enough, extra N can be fertilized.

% acidity
Common cattail grows well under acidic conditions ranging from a pH 5 to 7, but encounters problems if pH goes towards 3.5 \citep{brix2002typha}. Narrowleaf cattail seems to be able to deal with more extreme situations such as alkaline and saline situations. A pH of 9 can still be fine for Narrowleaf cattail.

% Maybe add extra column to table for difference between cattail species.

%Evapotranspiration can be between 105-150 cm per year, as reported in Poland \citep{dubbe1988production}.

\subsection{Peat moss}

% water level
In contrary to cattail, peat moss is a vulnerable species. Physical conditions should be within a delicate range. Peat moss grows best with the water level just below surface.  \citet{fritz2014paludicultuur} note that to optimize peat moss growth, the water table should be kept constant at 2-5 cm below ground level. However, as \citet{gaudig2014sphagnum} mention, it is also possible to grow peat moss on floating mats on the water.

% nutrients
\citet{Temmink2017196} show that peat moss grows well under high N input as long as pH is low enough and P and K inputs are high enough. 
% Note why that is important
N input was noted as ~ 30 kg per hectare per year and P was noted as between 1 and 3 kg per hectare per year.  The authors furthermore note that high nutrients can cause competitor vascular plant weeds such as \textit{Juncus effusus} or algae to grow. However, in this study these plants were removed, which proved useful for the peat moss to grow. \citep{wichtmann2016paludiculture} note that peat mosses in natural bogs grows best with low nutrient contents.

\citet{Temmink2017196} note that pH should be kept low, which can be done by making sure bicarbonate concentrations in waters are low enough. In the experiment, ditch water pH concentrations were between 4.8 and 6.0. Also, in other studies it is mentioned that peat moss grows best in acidic conditions \citep{wichtmann2016paludiculture}. Also, \citet{gaudig2005growing} found that peat moss performed significantly better at a pH of 3.2 compared to 4.5 and 8.0. 

\citet{fritz2014paludicultuur} note that it would be better to remove the top layer of a previously used agricultural peatland, so the trophic conditions are better accustomed to peat moss growth. That this procedure is ideal for peat moss growth was also mentioned by \citet{Temmink2017196}.

%What is the precise perfect stoichiometry?
\subsection{Azolla}

\textit{Azolla} is a water plant that needs full water cover. It can only survive in dry fields for a few days \citep{wagner1997azolla}. The optimal water depth is 5 cm, although it will survive deeper depths \citep{wagner1997azolla}. \citet{sabetraftar2013review} revealed that Azolla plants can only grow when it is floating on water. They furthermore say that Azolla plants can dry out if humidity is below 60\% but that Azolla can grow well between 55 and 83 \% humidity.  Waves and winds could be a cause for death of Azolla, since it would break the fronds (leaves). Cattail is noted as a protector of Azolla for wind and waves \citep{sabetraftar2013review}.  

\citet{lumpkin1980azolla} note that the most common nutrient limiting Azolla growth is P. The authors note that in Denmark, Azolla was found to thrive with a P concentration of 1.1 mg P/liter. Also, Iron (Fe) can become limiting, but in general 1 ppm iron is  sufficient for Azolla growth. However, the authors note that iron deficiency might stem from a high pH, that is because iron can then be precipitated. They acknowledge, that the more P is put into the water, the more N can be fixated. \citet{sabetraftar2013review} note that Azolla is known to not be affected by high N presence. In laboratory experiments, P concentrations of 0.06 ppm is sufficient, in field studies, they found P requirements of 0.3 - 1 ppm. Azolla plants can survive in salt concentrations between 160-380 mg/l  \citep{lumpkin1980azolla}. However, it was studied that high salinity can inhibit growth \citep{sabetraftar2013review, lumpkin1980azolla}. 

The optimum pH range is from 4.0 - 4.5, but Azolla can also survive in water with a pH from 3.5 - 10.0. However, vulnerability to pH is much affected by other factors. For instance, with high light intensity, optimal pH is 9-10 but with low light intensity, optimum pH is 6.0 at a temperature of 20 degrees \citep{wagner1997azolla}. Optimal pH levels according to \citet{lumpkin1980azolla} are in a range from 4.5 - 7.0. \citet{sabetraftar2013review} note that, in general, a pH between 4.5 and 7.5 is sufficient.



\chapter{GHG reductions by paludiculture}

% RQ: How can GHG emissions caused by peat decomposition in the Netherlands be reduced by paludiculture crops?
Peatlands are carbon stores. It is estimated that 1/3 of all carbon is stored in peatland ecosystems. However, current drainage practices have resulted in release of carbon from peat soils. The release of carbon in the form of GHG emissions triggers the atmospheric greenhouse effect resulting in a warmer climate. CH4 is a more potent gas than CO2 to warm the earth, but has a rather short duration time at earth. Also, N2O - an even stronger GHG gas - can be emitted from peat soils. The following section sheds light on processes behind emissions of GHG emissions from peatlands.

\section{Aerobic and anaerobic peat decomposition in the Netherlands}

% GHG emissions caused by decomposition in drained situation
Peat is formed by peatland ecosystems with water table levels close to the peat surface (fluctuating between -20 cm and 0 cm). On top of a peatland, plants capture carbon dioxide (CO2) through the process of photosynthesis. When these plants die, the dead plant material is available in the oxic top peat layer where it can immediately oxidize as CO2 back to the atmosphere. The 10-30\% risiding dead plant material is not directly oxidized and is transferred to lower anoxic peat layers. In these anoxic layers, anaerobic methanogenesis bacteria can decompose organic matter to Methane (CH4). Methane partially stays in the anoxic layers, and is partially transported back to the atmosphere. Depending on the height of the oxic peat layer, CH4 can be decomposed to CO2 and H2 by methanotrophic bacteria before it is emitted to the atmosphere. Plants with long inner tubes towards their roots (aerenchyma tissue) affect both CH4 release as well as CH4 oxidation since oxygen can enter through the tube and oxidize CH4 and CH4 can be transported through the tube to the atmosphere. 

In case of drained peat soils, water levels become so low that most CH4 is already oxidized before it is emitted to the atmosphere. However, in drained peat soils, organic substrate is provided to aerobic micro-organisms. These micro-organisms can decompose the organic matter that is stored in the peat while emitting CO2. Moreover, in this process, N2O is emitted to the atmosphere. After rewetting drained peat soils, water levels return to around the water table level. However, most of these rewetted peat soils have an agricultural background which resulted in eutrophication and altered pH levels. Both additional nutrients as well as altered pH have an adversive effect on initial CH4 emissions after rewetting. 

GHG emission factors and Global Warming Potentials (GWP) have been calculated for drained peatlands. These rates are used for the IPCC to calculate GHG emissions per country. Relevant for peat soils in the Netherlands, emission factors and GWPs were calculated for deep drained croplands, deep drained grassland with high eutrophication and for deep drained grassland with low eutrophication. GWPs are calculated based on emission factors of CO2-C, CH4-C, N2O-N and DOC, expressed as CO2-C equivalents and adjusted for duration times in the atmosphere. Overall, GWP potentials of temperate deep drained croplands.

\section{Peat decomposition rate differences in the Netherlands}
The rate of peat decomposition and following GHG emissions, depends, next to water level and nutrient conditions also on chemical and physical properties of peat soil \citep{}. For instance, some peat types have higher soil organic matter content and accordingly, more peat can be decomposed. Moreover, the presence and thickness of mineral soils below or above the peat layer can affect peat decomposition processes and thus affect GHG emission rates. In the Netherlands, peat soil classifications have been performed to distinguish different peat soils for their physical and chemical properties as well as mineral layer presence. The first step in the Dutch organic soil classiication is the distinction between peat ("veen") and peaty ("moerige") soils. Peat soils are classified as soils with 40 cm of peaty substance within the uppermost 80 cm of the soil. Peaty soils instead, have ... in the ...  The classification of peat soils is based on two base characteristics of the peat soil; (1) upper soil layer characteristics (main type) and (2) lower soil layer characteristics (sub type). 

In order to limit GHG emissions from peat decomposition in the Netherlands, it is useful to limit GHG emissions in peat types where current GHG emissions are highest. Since most peatlands in the Netherlands are drained, anaerobic peat decomposition is negligable. Thus, aerobic peat decomposition (oxidation) can be appointed as main contributor of GHG emissions from peat soils in the Netherlands. In order to locate areas with highest oxidation rates, the oxidation potential of different peat soil types is first to be determined.

Stouthamer has calculated indices for oxidation vulnerability of peat soils for various peat soil types in the province of Utrecht. These indices were used to map the spatial distribution of peat oxidation vulnerability in Utrecht. In order to calculate the indices, a method was developed. This method is based on three factors that were determined for all peat types: 

\begin{enumerate}
\item Profile type
\item Depth and thickness of mineral and peaty layers
\item Organic matter content of peaty material
\end{enumerate}

Five profile types (Figure ..) were distinguished, based on presence and vertical location of mineral soils:

\begin{enumerate}
\item Organic material
\item Organic material on top of mineral material
\item Mineral material on top of organic material
\item Mineral material with a layer of organic material
\item Mineral material

Each profile type was assigned an index (A) based on potential to oxidize organic matter (Table ..). A higher index corresponds to higher vulnerability to oxidize. Type 1 and 2 are more vulnerable to peat oxidation since the top layer is organic material. Moreover, type 1 and 3 are more vulnerable to peat oxidation since a mineral lower layer is devoid. Type 4 has organic material inclined in mineral layers and has thus a lower index value. Since no organic material is present in profile type 5, the index value is 0. 

Furthermore, a second index (B) was calculated based on the depth of the peaty layer. The closer the peat material is to the ground surface level, the higher the oxidation potential and therefore the index is. This index was calculated per 10 cm below ground surface level (Table ..). Moreover, an index for soil organic matter content (index C) was determined per 10 cm below ground surface level. This third index depends on the amount of SOM present in the soil layer. Therefore, for five categories, the SOM was calculated based on averages for the province (Table ..). Subsequently, for all peat soil types in Utrecht, these indices were determined based on horizont profiles. 

For each peat soil type in Utrecht, indices B and C were multiplied per 10 cm layer and summed up. The resulting factor was then multiplied with index A. To normalize the outcomes, the final step was to divide the number with the highest found index. In total, for 28 soil types, indices were determined. However, in the Dutch soil type map with size 1:50.000, a total of 63 soils with some organic material were classified. In order to obtain indices for all peat soil types in the Netherlands, two approaches were used. 

First of all, it was checked whether the indices used by Stouthamer could be used to estimate indices of other peat soil types. That was done by checking for similarity of indices between the two peat soil classification main and sub types. It was found that all main types (coded as aV, hV, pV, V, and zV) had similar indices (see Figure ..). Moreover, the deviations of sub types from the average maintype indice were in the same range (Figure ..). Therefore, for peat soil types that had a main type of V, hV, pV, V, or zV, indices could be calculated by taking the average indice for the main type and adding the average deviation of the sub type. As a result, for .. peat soil types, indices could be extrapolated. However, for two main type groups, no indices were yet calculated, namely the 'Waardveengronden' (code ..) and 'Gronden met veenkoloniaal dek' (code ..). Therefore, indices were built from scratch using the factorial method of Stouthamer. 

% Description of derivation index A,B and C from literature.

As a result, for 'Waardveengronden' respectively 'Gronden met veenkoloniaal dek' indices of .. and .. were found. In order to differentiate between sub types, standard deviations of sub types were added to the main indices. As a result .. more indices were found. 

% explanation of choosing indices for remaining peat soils

by applying paludiculture, it is most useful to apply paludiculture in places where the net GHG emissions from peat soils can be reduced most by paludiculture. Therefore, next to determine oxidation potentials in drained conditions, it is useful to determine peat decomposition potentials in paludiculture condition. However, that is not the scope of this study.

\section{Paludicultures possibilities to reduce GHG emissions caused by peat decomposition}

Various studies have been performed to assess GHG emission rates after implementing paludiculture. Most studies show an immediate decrease in CO2 emissions and an increase in CH4 emissions. Overall, for IPCC emission factors in temperate zones, rewetted peatlands have a GWP of .. That is a decrease of .. compared to a drained situation.
 
% impact of rewetted peatland on GHG emissions
\citet{van2013rewetting} studied the effect of a rewetted peatland on net GHG emissions expressed in CO\sous{2} equivalents. The study showed that CO\sous{2} emissions decreased drastically and CH\sous{4} emissions increased. However, the net effect of rewetting the peatland with water table of 4 cm above the ground on emissions of CO\sous{2} equivalents was positive with approximately 1.0 kg m\sur{-2} less CO\sous{2} equivalents. Still, GHG emissions in terms of CO\sous{2} were as high as 2.0 - 3.0 kg m\sur{-2} in the rewetted condition.

A meta study was performed regarding water table level influencing CO2 equivalents emissions from peatlands. The result show a steady decrease from 40 ton CO2-e per hectare per are at -100 to 10 ton CO2-e at -20 cm. When water levels get higher than 0 cm, emissions increase again to 15 tons, but these emissions to not go up steadily anymore \citep{wichtmann}. In a presentation at the Paludiculture conference in Greifswald, new research was published and that as a rule of thumb, for each 10 cm increase in water level, 5 ton CO2-e per hectare per year can be decreased.

% biomass harvest

% top soil removal
An experiment with top soil removal in the Netherlands, showed a reduction of 99\% in methane emissions when rewetting the peatland \citep{harpenslager2015rewetting}. The results shows the potential of topsoil removal prior to rewetting for reducing GHG emissions. 
%K Explain why is this

% peat moss carbon balances
In Germany, carbon balances of establishing peat moss plots were made \citep{gunther2017greenhouse}. The authors found that methane emissions declined significantly over the study time of two years. The production strips of the peat moss  species were found to be net GHG sinks of 5 to 9 ton ha\sur{-1} a\sur{-1} (in CO\sous{2} equivalents). However, ditches were net carbon sources with release of about 11 ton ha\sur{-1} a\sur{-1} CO\sous{2} equivalents. Still, peat moss farming on bogs were found to be less emitting compared to low-intensity dairy farming. Moreover, the authors note that with peat moss farming, an alternative is provided for mineralisation of pristine peat areas for horticulture substrates.
%K explain more 
 
% cattail carbon balances
It is known that plants possessing aerenchuma such as cattail, are methane emitters \citep{wichtmann2016paludiculture}. A study on broadleaf cattail showed that the GHG balance after the first year of implementation was already close to climate neutral \citep{guntherghgtypha}. In the study, the authors compared a plot with harvest with a plot without harvest. Harvesting the cattail proved to reduce net GHG emissions slightly. \citet{wichtmann2016paludiculture} note that 'renewed biomass growth after cutting supports net carbon sequestration contributing to a decrease in GHG emissions'. 

\section{Water management options to reduce peat decomposition in the Netherlands}
In order to implement paludiculture to reduce GHG emissions from peat decomposition, water table levels need to be maintained around surface level (depending on specific crop - see Chapter ..). Therefore, water management changes need to be made locally. Different water management options are possible when implementing paludiculture. Currently, water table levels are maintained at catchment (peilvak) scale. These catchments and their functioning are the responsibility of Dutch Water Boards (Waterschappen). For paludiculture to reduce GHG emissions from peat soils optimally, catchments are chosen that lie within peat areas.

Paludiculture can be applied using whole catchments or using parts of catchments. When using a whole catchment for paludiculture, GHG emission proportion will be changed from CO2 dominated towards CH4 dominated emissions. Depending on water tables above the ground, CH4 emissions differ. Therefore, elevation of the soil plays a role in the GHG emissions that occur after paludiculture implementation. When using a part of a catchment for paludiculture, GHG emission proportions will reduce from CO2 dominated to both CO2 and CH4 emissions, depending on water levels above/below the soil. If a catchment is fully peaty, CO2 emissions are stopped in the rewetted part of the catchment and reduced in the dryer parts of a catchment since water level tables have increased. CH4 emissions will increase in the wetter part of the catchment but are devoid in the dryer part of the catchment. If a catchment is only partially peaty, it is important to make sure that the peaty part of the catchment is rewetted. Therefore, the peaty part preferably has a lower elevation compared to the non peaty part. 

Note that sub soil drainage (onderbemalingen) areas may be suitable for paludiculture. Sub soil drains are installed to lower water level tables locally. Water table levels in these areas are often easily heightened and are normally rather small, allowing for small plots of paludiculture. 

\chapter{Sustainability issues}

\section{Additional functions of paludiculture}

\subsection{Economic factors}
% incomes of farmers versus paludiculture
First of all, the question is whether paludiculture is a economically viable alternative to current agricultural practices. The areas with the most mineralizable peat are located in the Western and Northern part of the Netherlands, see chapter \ref{ch:optloc}. In these areas, the main agricultural use is intensive dairy farming. Some studies have shown the potential benefits of paludiculture versus intensive dairy farming. 

With intensive dairy farming, the average net income of a farmer is 565 euro per year \citep{van2013rewetting}. As an addition an intensive dairy farmer receives on average 400 euros of subsidies from the governments per hectare per year \citep{van2013rewetting}. At the moment, one of the most profitable paludiculture crops is peat moss, with which the net income is calculated as 2.650 euro per hectare per year \citep{van2013rewetting}. However, a first investment of 23.300 euro would be needed \citep{van2013rewetting}. As of 2017, no subsidies are given to paludiculture farmers yet. Note that land prices of agricultural lands in peat areas are between 50.000 and 60.000 euro per hectare \citep{}. Starting a paludiculture business when someone is not farming yet, might thus be difficult.

% accounting for other ecosystem services
If ecosystem services costs and benefits would be accounted for, paludiculture would be more economically viable. For example, with a functioning carbon credits system of 35 euro per ton reduced GHG emissions, a peat moss farmer could earn 466.67 euros extera per hectare per year. Moreover, \citet{van2016dalende} calculated that costs of damage to buildings and infrastructure as a result of peat decline will be in total 1 billion euros till 2050, which would be ~ 30 million euros per year. If other ecosystem services which are presented below would be accounted for, benefits of paludiculture would be higher.

\subsection{Ecological factors}
% Low lying nature conservation
In the Netherlands, a number of nature conservation areas lie in peatlands. These areas often have a higher elevation than their surroundings. That causes seepage from water away from the nature area, resulting in drougths and high costs to keep the water in. Paludiculture as buffers around nature areas are mentioned as a good alternative to prevent seepage from nature areas, and also to function as a bufferzone for aquatic fauna species such as the aquatic warbler \citep{van2013rewetting}. Besides acting as a nature bufferzone, paludiculture areas can be used as water storage bufferzones, in case of extreme rainfall, which is expected to happen more frequently in the future. 

% Nutrient filtering
Surface waters in the Netherlands are often eutrophied due to nutrient runoff after intensive fertilizer and manure application in agricultural fields. In special, N and P are abundantly available in surface waters, causing algae and other plants to bloom. As a result, waters can be deprived of oxygen, resulting in a (often undesirable) change of aquatic fauna \citep{waajen2014eutrophic}. Paludiculture crops have shown to be able to thrive under these high nutrient conditions (see chapter \ref{ch:abiot}). Waters can be cleaned by paludiculture crops, which is already done in waste water treatment plants \citep{}. Depending on the nutrient concentrations, wetland species can extract more than 300 kg N and 30 kg P per hectare per year \citep{Land2016}. 

%land2016


\section{Bottlenecks with paludiculture}

% Meadow birds
A topic that deserves special attention is the declining meadow bird populations in the Netherlands. Some of these species, such as the Godwid, has its main habitat in the Netherlands. Therefore, the Netherlands has a special opportunity to protect these species. The main reason for the decline in meadow bird populations in the last decades has been the more intensive harvest of grass on meadows in the Netherlands. However, rewetting of drained peatlands and implementing paludiculture would most probably imply even less space for meadow birds (personal note - Alex Schotman). Moreover, important meadow bird sites are often located in peat meadows, as is depicted on the map. However, note that other biodiversity such as protected sundew, aquatic warbler and orchids can be improved as a result of paludiculture. Furthermore, some paludiculture species might be good for meadow birds, if the water level can be maintained a few centimeters below surface (personal note - Alex Schotman). Maintaining the water below surface is exactly what is needed for peat moss farming.

% Cultural landscape
Cows grazing on Dutch meadows is something that many Dutch people familiarize themselves with. Moreover, according to a partitioner at a 'veenmiddag' in the province of Noord-Holland, 'the peat meadows are a special cultural landscape with an openness that has not altered over decades \citep{Dosker2017}. In special, the partitioners wanted to keep the typical ditch structures as it has always been. These ditch structures are known as the "Slagenlandschap". Another attendent, someone who lived in a city mentioned that it is not necessary to maintain the landscap as it is, but more to look at opportunies such as energy landscapes close to home \citep{Dosker2017}. All in all, the attendents agree that peat loss should be stopped and that paludiculture is a welcome alternative, at least in some places \citep{Dosker2017}


\section{Temporal issues regarding paludiculture}

The Dairy farming industry has formed the Dutch policies, values and customs of the Dutch regarding usage of peat meadows. Therefore, farming of paludiculture would therefore be a major transition for the current land users. A few key issues that prevent to take action to transform to paludiculture will be discussed below.

First of all, paludiculture is not high on the political agenda, yet. Subsidies for paludiculture practices are not at play. Sometimes, regulations are even opposing paludiculture practices. For example, in Germany, cattail may not be farmed since it is listed as a protected species. 

Secondly, Dutch farmers and agrologists are not convinced about making a transition to paludiculture. There is not enough financial security to make the transition to be a paludiculture farmer. However, paludiculture could be an addition to the current practices farmers perform. The nutrient runoff from their farmyard could be used for paludiculture crops, and paludiculture crops could be an addition to fodder for the cows. 

Thirdly, markets are currently underdeveloped. Paludiculture breeding to obtain the most efficient, the most robust plants is in its infancy. Since large scale production is not taking place, factories are not built and usage of the products is too expensive. 

Although the current market, politics and farmers are not ready yet for large scale paludiculture implementation, small steps should be taken already to allow for future sustainable paludiculture practices. First of all, better sooner than later, the subsidy system for paludiculture should be updated so that farmers have more financial security. 

Secondly, each location is different, and therefore custom made solutions should be made, depending on the local interests at play. Whether these interests are protection of certain bird species and therefore taking a lower density of cattail products, maintaining the "Slagenlandschap" cultural look of the landscape by farming peat moss instead of cattail, nutrient filtering at farmyards or highly economic production where other social or ecological factors are of less importance.

\chapter{Spatial opportunities of paludiculture}

\section{Spatial distribution of optimal locations in the Netherlands to reduce GHG emissions caused by peat decomposition}

\subsection{Abiotic conditions}

In the Netherlands, surface water quality is measured in many places. In 2013, more than 10.000 measurement locations were sampled. For the parameters used for this study, the number of measurement locations per parameter and the number of locations in peat areas can be found in table \ref{tab:abioticmeasurements}.

In table .., the number of locations that meets the conditions of each paludiculture crop can be found. Moreover, Figure .. presents per province the suitability per parameter. 

\subsection{Vulnerable areas for peat decomposition}

For 28 peat soils in the Netherlands, mineralization potentials were calculated by \citep{stouthamer2008toelichting}.  In this study,  the list of mineralization potentials was extended to include also the other peat types in the Netherlands. The results are presented in Annex \ref{ann:indicesused}. Furthermore, the different peat types were categorized in four categories of mineralization potential, which can be seen in Table \ref{tab:minpotcategories}. 
All in all, the provinces of Friesland, Zuid-Holland and Noord-Holland have the highest proportion of very high peat mineralization soils.

\ref{ann:mineralmap}. Table \ref{tab:sizesmincatprovinces} shows the area size of the different mineralization categories in the different provinces in the Netherlands. Furthermore, in Figure \ref{fig:mineralpotnl} a map can be found portraying per province the area size of each mineralization category. The areas with categories of mineralization potentials are summarized in Table \ref{tab:sizesmincat}.

\subsection{Catchments with a lot of peat}

For five Water Boards, 313 catchments were found that have some peat soil. Table .. lists per waterschap, the number of catchments that have some peat soil. These catchments are displayed in Annex \ref{ann:catchwithpea}. Moreover, the percentage of peat soils versus non peat soils in each remaining catchment was calculated. Following, these percentages were categorized in four groups. For the 5 Water Boards, the results are shown in Table \ref{tab:catcatchwhole} Also the contribution of each category to the total catchments with peat is shown. Figure \ref{fig:peatproportioncatch} displays a map with the different catchments and the amount of peat that can be found in each catchment.

\section{Spatial opportunities regarding usage of elevation differences in catchments to reduce GHG emissions caused by peat decomposition in the Netherlands}

Therefore, elevation differences of peaty versus non peaty parts of eight catchments were tested. The names, the sizes and the amount of peat in the catchment can be found in Table ..

For all catchments, mean heights were calculated, as well as heights of the peaty and non peaty part of the catchment. Together with the difference, the results are presented in Table {tab:heigthcatchpart}. All peaty parts of catchments have a lower elevation than the non peaty parts with an average difference of 0.32. 


\section{Spatial co-functions and bottlenecks regarding paludiculture in the Netherlands}



\chapter{Discussion} \label{ch:con}

The conclusions that can be drawn from this study are as follows:

\begin{itemize}
\item The current abiotic situation is positive for cattail and azolla farming regarding pH, N and P. Peat moss has few places where the current pH is suitable. However, fertilizer (manure) application or other management options might help to create more suitable spaces.
\item The Netherlands has 130.678 hectares of highly mineralizable peat, which is mostly situatied in Friesland and Zuid-Holland. 
\item 313 catchments with a total size of .. hectares in five water boards were found suitable to reduce peat mineralization. Of these catchments, 44.11 \% had more than 75 \% of peat in their catchments. Therefore, these catchments contribute most to GHG emissions from Dutch peat soils.
\item Of the eight studied catchments, mean elevation was lower in the peaty part of the catchment with an average difference of 0.32 cm. Therefore, it can be concluded that rewetting these eight studied catchments could help to reduce GHG emissions.
\item Paludiculture markets are in development and thus are not proved financially solid. However, estimations hold promise and if ecosystem services (in special carbon storage) would be accounted for, paludiculture is more profitable than intensive dairy farming.
\item Ecologically, paludiculture provides interesting co-benefits such as nature drought prevention, water storage, nutrient filtering and improving biodiversity. Special attention should be given to meadow birds, since these birds often inhabit peat areas.
\item The current dairy farming practices are valued preciously by Dutch people. However, the problems regarding peat decline are acknowledged and people are open to paludiculture in some places. 
\item Peat decline affects people and the planet, now and in the future, here and in other places in the world. Action should be taken now, but is hampered by (1) underdeveloped markets, (2) lack of political support and (3) unconvinced farmers and agrologists. 
\item Steps that should be taken now to allow for sustainable future use of peat meadows in the Netherlands are (1) implementation of a proper subsidy system that gives security to transitioners and (2) take local situation into account by defining a solution that is suitable for that location.
\end{itemize}

Some methodological issues are presented below:
\begin{itemize}
\item Many analyses in these study are based on soil type data that was provided by Wageningen Environmental Research on a 1:50.000 scale. Some of the data that were used for creating the soil map are older than ten years. However, only recently, the map was updated for the peaty parts of the Netherlands. 
\item Mineralization indices were obtained from \citep{stouthamer2008toelichting}, without knowing whether these indices were created only for Utrecht peat soils and not to be extrapolated to other parts of the Netherlands. However, this study tried to give an holistic overview of the situation. For this, the method that was used is sufficient.
\end{itemize} 

Some new ideas for research:
\begin{itemize}
\item What exact GHG reduction can be achieved by rewetting (parts) of catchments?
\item Based on new elevation data (AHN3), what exact subsidence has taken place in the Netherlands?
\end{itemize}

\chapter{Conclusion} \label{ch:con}


